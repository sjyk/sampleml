\section{Approximating the Optimal Distribution}\label{sampling}
This section describes our solution to problem of constructing and maintaining an inexpensive estimate of the optimal distribution.

\subsection{Avoiding Regression}
One way to approximate this distribution is to learn a function $e(\cdot)$ mapping $(x_{clean}, y_{clean})$, based on the data that we have cleaned.
However, learning this function can be very expensive and is not guaranteed to be reliable.
This is a high-dimensional regression problem which may have to learn a very complicated relationship between dirty and clean data.
Such an estimator has a cold start problem, where if we have cleaned a very small number of errors, the estimator will be inaccurate.
This problem becomes increasingly worse in higher dimensions as we need more data for an accurate estimate.
We take an alternative approach, where we try to exploit what we know about data cleaning to produce an estimate for groups of similarly corrupted records.

\subsection{Error Decoupling}
We come back to the insight of the last section, where error detection is often much easier than error repair.
For many types of data error, we can return a subset of corrupted attributes and in turn a subset of features that are corrupted; all without cleaning the data.
Recall, that when we formalized the error detection problem, we ensured that associated with each $r \in R_{dirty}$ is a set of errors $e_r$ which is a set that identifies a set of corrupted columns.
We will show how we can use this property to construct a coarse estimate of the clean value.
The main idea is that if we can calculate average changes for each feature, then given an uncleaned (but dirty) record, we can add these average changes to correct the gradient.

Let us formalize this intuition.
Instead of computing the actual gradient with respect to the 
true clean values, let us compute the conditional expectation given that a set of features and labels $f_r$ are corrupted:
\[
p_i \propto \mathbb{E}(\nabla\phi(\theta_{(t)}^Tx_{clean},y_{clean}) \mid f_r)
\]
What we mean by corrupted features is that:
\[
i \notin f_r \implies x_{clean}[i] - x_{dirty}[i] = 0
\]
\[
i \notin f_r \implies y_{clean}[i] - y_{dirty}[i] = 0
\]
So basically, if most of the features are correct, it would seem like the gradient is only
incorrect in one or two of its components.
The problem is that the gradient $\nabla\phi(\cdot)$ can be a very non-linear function of the features that couple features together.
For example, let us look at the gradient for linear regression:
\[
\nabla\phi(x,y,\theta) = (\theta^Tx - y)x
\]
We see that it is not possible to isolate the effect of a change of one feature on the gradient.
To understand why we care so much about the decoupling consider the following example:
\begin{example}
Suppose that we have just cleaned the following records represented as tuples with their corrupted feature set: ($r_1$,$\{1,2,3\}$), ($r_2$,$\{1,2,6\}$).
Then, we have new record ($r_3$,$\{1,2,3,6\}$). 
We want to be able to use the cleaning results from $r_1,r_2$ to estimate the gradient in $r_3$.
\end{example}
Decoupling allows us to treat errors conditioned on each feature independently.
Alternative techniques such as taking the average change the gradient, without this property, would require to condition on every distinct set of corrupted features which can be combinatorially large. 

\subsection{Linear Approximation}
We can approximate the gradient in such a way that we can do this.
This approximation represents a linearization of the errors.
We can take the expected value of the Taylor series expansion around the dirty value.
If $d$ is the dirty value and $c$ is the clean value, the Taylor series approximation for a function $f$ is given as follows:
\[
f(c) = f(d) + f'(d)\cdot(d-c) + ...
\]
If we ignore the higher order terms, we see that the linear term $f'(d)\cdot(c-d)$ decouples the features.
We only have know the change in each feature to estimate the change in value.
In our case the function $f$ is the gradient $\nabla\phi$.
So, the resulting linearization is:
\[
\nabla\phi(\theta^Tx_{clean},y_{clean}) \approx \nabla\phi(\theta^Tx,y) + \frac{\partial}{\partial X}\nabla\phi(\theta^Tx,y)\cdot (x - x_{clean}) \]
\[+ \frac{\partial}{\partial Y}\nabla\phi(\theta^Tx,y)\cdot (y - y_{clean})
\]
When we take the expected value:
\[
\mathbb{E}(\nabla\phi(\theta^Tx_{clean},y_{clean})) \approx \nabla\phi(\theta^Tx,y) + \frac{\partial}{\partial X}\nabla\phi(\theta^Tx,y)\cdot \mathbb{E}(\Delta x) \]
\[+ \frac{\partial}{\partial Y}\nabla\phi(\theta^Tx,y)\cdot \mathbb{E}(\Delta y)
\]
So the resulting estimation formula takes the following form:
\[
\approx \nabla\phi(\theta^Tx,y) + M_x \cdot \mathbb{E}(\Delta x) + M_y \cdot \mathbb{E}(\Delta y) 
\]
Recall that we have a $d$ dimensional feature space and $l$ dimensional label space.
Then, $M_x = \frac{\partial}{\partial X}\nabla\phi$ is an $d \times d$ matrix, and $M_y = \frac{\partial}{\partial Y}\nabla\phi$ is a $d \times l$ matrix.
Both of these matrices are computed with respect to dirty data, and we will present an example.
$\Delta x$ is a $d$ dimensional vector where each component represents a change in that feature and $\Delta y$ is an $l$ dimensional vector that represents the change in each of the labels. 

Let us return to the linear regression example, where the gradient is:
\[
\nabla\phi(x,y,\theta) = (\theta^Tx - y)x
\]
It we take the partial derivatives with respect to x $M_x$ is:
\[
M_x[i,i] = 2x[i] + \sum_{i \ne j} \theta[j]x[j] - y 
\]
\[
M_x[i,j] = \theta[j]x[i]
\]
Similarly $M_y$ is:
\[
M_y[i,1] = x[i] 
\]
In the appendix, we describe the matrices for common convex losses \reminder{TR}.

\subsection{Maintaining Decoupled Averages}
This linearization allows us to maintain per feature (or label) average changes and use these changes to center the optimal sampling distribution around the expected clean value.
We know how to estimate $\mathbb{E}(\Delta x)$ and $\mathbb{E}(\Delta y)$.
\begin{lemma}[Single Feature]
For a feature $i$, we average all records cleaned that have an error for that feature, weighted by their sampling probability:
\[
\bar{\Delta}_i = \frac{1}{K}\sum_{j=0}^K (x[i]-x_{clean}[i])\times \frac{1}{p_j}
\]
Similarly, for a label $i$:
\[
\bar{\Delta}_i = \frac{1}{K}\sum_{j=0}^K (y[i]-y_{clean}[i])\times \frac{1}{p_j}
\]
\end{lemma}

Then, it follows, that we can aggregate the $\bar{\Delta}_i$ into a single vector:
\begin{lemma}[Delta vector]
Let $\{1..,i,...,d\}$ index the set of features and labels.
For a record $r$, the set of corrupted features is $f_r$.
Then, each record $r$ has a d-dimensional vector $\Delta_r$ which is constructed as follows:
\[
 \Delta_r[i] = \begin{cases} 0 & i \notin f_r \\ 
\bar{\Delta}_i & i \in f_r
\end{cases} 
\]
\end{lemma}

With the above theorem, we finally have an approximation to our sampling weights: 
\[p_{r}\propto\|\nabla\phi(x,y,\theta^{(t)}) + M_x \cdot \Delta_{rx} +  M_y \cdot \Delta_{ry}\|\]



