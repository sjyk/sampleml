\section{Imperfect Partitioning}\label{imperfect}
An important aspect of our gradient update is partitioning the dirty and clean data.
We aggregate an average gradient from both subpopulations when making our update.
When our partitioning is perfect, at least all of the dirty data is selected in $R_{dirty}$, this estimate in unbiased.
However, in some cases, characterizing this partioning can be difficult and it may be impossible to ensure this condition unless $R_{dirty} = R$ causing a loss in efficiency if errors are sparse.  
In this section, we explore the case when the partitioning is imperfect.

\subsection{Why do we partition?}
Partitioning serves two purposes: (1) it reduces the variance in the gradient update, and (2) it increases the fraction of actually dirty records in the candidate batch.
The first objective follows directly from the analysis in the previous section (Section \ref{analysis}).
The second objective is best understood in the context of crowdsourcing.
If we have a crowdworker clean records, we will have to pay them for the task whether or not the record required cleaning.
Partitioning the data ensures that the only records cleaned are dirty records.
Without partitioning, cleaning rare errors might be very wasteful. 

\subsection{Revised Formulation}
Even in this setting, our framework is still useful, however we have to modify the formulation of the problem.
Instead of assuming that we know which records are dirty in advance, we make the assumption that we know which attributes are dirty in advance.

\noindent\textbf{Error Repair: } When an example $(x,y)$ is cleaned, for each dirty attribute $a$ cleaned becomes a training example for an error classifier $\kappa_a$. This classifier, e.g., Support Vector Machine, that learns which points are dirty and which points are cleaned based on the results of operation.

\noindent\textbf{Error Detection: } To select $R_{dirty}$, we select the set of records for which any of the error classifiers $\kappa_a$ give a positive error classification.
Many types of classifiers allow users to tradeoff precision and recall.
In other words, we can also select any record within some level of confidence of the classification margin.
For an SVM, we may only classify a point as clean if it is sufficiently far from the margin.
Or for Logistic Regression, we may do so if its class likelihood is over 80\%.

\subsection{Revised Algorithm}
The general case algorithm is harder to analyze and thus we rely on empirical performance on real data.
\begin{enumerate}[noitemsep]
\item Initialize with $\theta^{(0)}$ as the dirty model, $T$ iterations, with a batch size $B$
\item Initialize all $\Delta = 0$
\item Initialize $R_{dirty} = R$
\item For rounds i=1...T
\begin{enumerate}
	\item Sample $B$ candidate dirty data points with probabilites as described.
	\item Apply data cleaning to the sample of data.
	\item Apply weighted gradient descent to update the model.
	\item Update $\Delta$ for each feature.
	\item Update classifiers $\kappa$, and apply the classifier to prune the set $R_{dirty}$ to records clean up to sufficent confidence.
\end{enumerate}
\item Return $\theta^{(T)}$
\end{enumerate}
