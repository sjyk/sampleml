{\noindent \normalsize \bf Dear SIGMOD Chair and Referees: }

\vspace{.5em}

We thank the reviewers and chair for the helpful feedback on our paper. 
We addressed all of the concerns and included references to the revised text. 
To summarize the major changes:

\begin{enumerate}
\item Sections 1 and 2 clarify the contributions of \sys and its relationship to related work (e.g., \cite{gokhale2014corleone, DBLP:journals/pvldb/YakoutENOI11, yakout2013don}).

\item In Section \ref{dmodel}, we formalize the definition of dirty data and the data cleaning model.

\item Section \ref{s:usecase} presents a running example that is referenced in each technical section (Examples \ref{archex}, \ref{upex},\ref{detex1},\ref{detex2},\ref{estex}).

\item In Section \ref{statements}, we provide problem statements for the two subproblems addressed by \sys.

\item Section \ref{arch} presents a revised system architecture that first emphasizes the essential components for correctness, and then highlights optional optimizations. 

\item We include references to all of the related work suggested by the reviewing committee \cite{whang2014incremental, papenbrock2015progressive, gruenheid2014incremental, DBLP:journals/pvldb/YakoutENOI11, yakout2013don, heise2014estimating}.
\end{enumerate}


\subsection*{Overview} 
Data cleaning is often applied before featurization and predictive modeling to provide clean training data.
Errors in the training data can degrade model quality by masking or introducing spurious relationships between features.
Unfortunately, data cleaning is often a manual and time consuming process, and may be impractical for very large datasets.
\sys is a framework that allows users to train accurate models without cleaning the entire dataset. 
\sys couples data cleaning with model training to leverage information from the model to identify records that maximally improve accuracy.
The paper shows that naive integration of data cleaning and model training can lead to convergence problems, and we present a novel framework for training on partially clean data.
Specifically, it applies: (1) a gradient-based update derived from small batches of cleaned data, (2) weighted sampling to carefully select these batches, (3) optimizations to select records that are most likely to be dirty.

\sys is similar to active learning as both techniques seek to improve the efficiency of expensive manual or crowdsourced data cleaning.
Consequently, the reviewers asked whether existing active learning approaches could be applied to the problem of prioritizing data cleaning for user-specified modeling tasks. 
In terms of correctness, active learning is not designed for this problem setting where training happens on mixtures of clean data and not-yet-cleaned dirty data.
Due to the well-studied Simpson's paradox, this can lead to severe inaccuracy (Section \ref{correctness}).
Even if the correctness problem is addressed, the new problem setting leads to several new opportunities for optimizations, which empirically improve on typical active learning criteria by up to an order of magnitude (Section \ref{eval}).

\vspace{0.5em}

\subsection*{Meta Review Details} 

\noindent\noindent \textbf{M1. There should be a formal vocabulary introduced early on. The exact idea of ``dirty" here can be hard to follow: what is the exact error type(s) that the system is intended to clean?}

\vspace{0.5em}

We added a section that clarifies that our system supports data cleaning operations that can be represented as record-by-record mappings (Section \ref{dmodel}).
Formally, there exists a function (implemented via human or algorithm) that given a dirty record will return a unique clean record.
This does not include errors that simultaneously affect multiple rows such as duplication or schema transformation problems.
%Details are provided in Response \textbf{R2.1}.

\vspace{0.5em}

\noindent\textbf{M2a. Sections 5-7 are the technical core of the paper, and appear formal at the expense of aiding understanding.}

We have revised the technical sections of the paper to improve readability by moving derivations to the appendix and including three new examples.

\vspace{0.5em}

\noindent \textbf{M2b. They appear to implement something that resembles active learning or bootstrapping, except inside the gradient descent loop. The motivation of some of this is not clear; is it necessary to integrate with the gradient descent?} 

Yes, the integration with gradient descent is necessary as it allows for provable guarantees about the model's convergence.
We have revised the paper to intuitively motivate this problem in Section \ref{intuit}, formally describe the problem in Section \ref{updp}, and simplified the presentation of the gradient-based update solution in Section \ref{geod}.

\vspace{0.5em}

\noindent\textbf{M2c. This is not how most active learning methods are implemented. Is it possible to implement these approaches in a way that is orthogonal to the SGD algorithm? The current writeup entangles some of these design choices.} 

We have revised the paper to decouple two subproblems: (1) the correctness problem of how to update a model after cleaning, and (2) the efficiency problem of how to prioritize cleaning using the model. 
Problem (1) is addressed with SGD.
The solution to problem (2) is presented orthogonal to the update problem.
We clarify that any sampling algorithm that ensures that all dirty records have a non-zero sampling probability can be applied.
We also have re-organized the paper to isolate optimizations from the essential components of \sys.
In the revision, Section \ref{opti} describes optimizations that improve the convergence rate of the system.
%We describe a number of cases when these optimizations are possible.

\vspace{0.5em}

\noindent\textbf{M3. In general, the distinction between an ``architecture" and an ``algorithm that fits into the architecture" is quite unclear. The problem with SGD/Active Learning above is one example.}

We have significantly revised the architecture section of the paper.
We first separate the formal problem statement (Section \ref{statements}) and system architecture (Section \ref{arch}).
The architecture section now describes the essential data flows of the system and is orthogonal to the solutions of the problems described in Section \ref{statements}.
We also differentiate the essential components of the architecture and the optional instance-specific optimizations.
The new architecture would apply even if the model update problem was addressed with a different algorithm (i.e., not SGD).
%We also clearly identify the user inputs in Section \ref{uinp}.

\vspace{0.5em}

\noindent\textbf{M4. The paper, and especially the technical sections, would benefit enormously from a detailed running example showing how the algorithm works}

We have added examples to each of the technical sections based on our running example of an analyst using an SVM for fraud detection. 
Section 4 (Architecture) describes an intuitive end-to-end application (Example \ref{archex}).
Section 5 (Update Problem) describes how updates are propagated and calculated (Example \ref{upex}).
%Section 7.1 (Detection) contains two examples for how the two different types of detectors can be used (Examples \ref{detex1} and \ref{detex2}).
Section 7.2 describes how to apply the optimizations to this example.

\vspace{0.5em}

\noindent\textbf{M5. Some connections to related work that combines machine learning and data cleaning should be made. See the other reviewers' comments.}

We have added Section \ref{alrw} to connect \sys to related work that applies machine learning to data cleaning.
This was a source of significant confusion in the initial submission, and we have clarified the key differences.
We have also revised our related work section to highlight the suggested references to progressive data cleaning and entity resolution~(Section \ref{rw}).
%Details are provided in Response \textbf{R1.4} and Response \textbf{R2.2}.

\subsection*{Review 1 Details} 

\noindent\textbf{R1.1: At first, the problem seems a bit too specialized. The abstract is too loaded with technical terms and a turn-off. This is then mitigated in the introduction. \\
As mentioned above, the abstract is (to me) overly technical and did not make me curious. I did not know off the bat what a convex loss model is, what importance sampling is, etc.}

\noindent We revised the abstract to be more accessible:

\emph{Dirty data, including missing, incorrect, or inconsistent values, is an important challenge in data analytics.
Predictive models, such as regression and classification, are increasingly popular and can be highly sensitive to dirty data.
Although error can be mitigated through data cleaning, it is often very time consuming.
This paper explores techniques to train accurate models without having to clean the entire dataset.
The challenge is that models trained on partially cleaned data can be arbitrarily incorrect requiring a new algorithm for incrementally updating results given newly cleaned data.
We also design sampling algorithms that leverage the structure of downstream models to prioritize cleaning those records likely to affect the results.
We focus on a popular class of models called convex loss models (e.g., linear regression and SVMs).
The key insight of \sys is that data cleaning can be applied simultaneously with incremental optimization allowing for progressive cleaning while preserving provable properties.
Evaluation on four real-world datasets suggests that for a fixed cleaning budget, \sys returns more accurate models than uniform sampling and Active Learning when corruption is systematic and sparse.}

\vspace{0.5em}

\noindent\textbf{R1.2: Poor embrace of the duplicate detection problem (see details below). Your model of the cleaner seems to preclude any duplicate detection, which certainly cannot happen on individual records. Also you extension for a set of record does not fit the problem of duplicate detection. This is in contrast, for instance, to your ER example in the second column of that page. Appendix A.1 is misleading here, as you mention with Example 7 ``in entity resolution problems..." but do not actually address that problem in the example. Fixing some common inconsistency is not entity resolution.}

We apologize for the confusing terminology and have revised our paper to clarify that we do not address record-level deduplication and entity resolution.

\vspace{0.5em}

\noindent\textbf{R1.3: Cheated by using a narrower font than required. Will have trouble with camera ready copy if publisher insists on proper font.\\
- I would not use ``overview" as a verb...
- 3.2: the detector select -> the detector selects
- 4.3: Wrong quotation marks around ``learning".
- QED symbols on page 8 are ugly when placed directly after formula. 
- References need a clean up. Just as an example: Venue is missing for [24], year is mentioned 3 times for [8], [11], etc. Page numbers appear sporadically.}

\noindent We have addressed all of the formatting and copy editing issues.

\vspace{0.5em}

\noindent\textbf{R1.4:There is some related work specifically addressing progressive/incremental entity resolution. You might want to point your readers to this.
\\E.g.
\\- Incremental entity resolution on rules and data, Whang et al. VLDB Journal 2014
\\- Progressive duplicate detection, Papenbrock et al., TKDE 2015
\\- Incremental record linkage, Gruenheid et al., PVLDB 2014
\\- Another work that is related is ``Estimating the Number and Sizes of Fuzzy-Duplicate Clusters" by Heise et al. CIKM 2014, which also incrementally cleans samples of data to predict in this case the number of record matches.}

Thank you for highlighting these references, and we have included them in our related work:

\emph{When data cleaning is expensive, it is desirable to apply it \textbf{progressively}, where analysts can inspect early results with only $k \ll N$ records cleaned.
Progressive data cleaning is a well studied problem especially in the context of entity resolution \cite{altowim2014progressive, whang2014incremental, papenbrock2015progressive, gruenheid2014incremental}.
Prior work has focused on the problem of designing data structures and algorithms to apply data cleaning progressively.
This is challenging because many data cleaning algorithms require information from entire relations.
Over the last 5 years a number of new results have expanded the scope and practicality of progressive data cleaning~\cite{mayfield2010eracer, DBLP:journals/pvldb/YakoutENOI11, yakout2013don}.
\sys explores the statistical implications of using progressive data cleaning before high-dimensional predictive modeling.
\\
SampleClean~\cite{wang1999sample} applies data cleaning to a sample of data, and estimates the results of aggregate queries.
Sampling has also been applied to estimate the number of duplicates in a relation \cite{heise2014estimating}. 
Similarly, Bergman et al. explore the problem of query-oriented data cleaning \cite{DBLP:conf/sigmod/BergmanMNT15}, where given a query, they clean data relevant to that query. 
Existing work does not explore cleaning driven by the downstream machine learning models studied in this work.}

\vspace{0.5em}

\noindent\textbf{- Page 1, last paragraph in column 1 reads as if reference to [3] is a reaction to the work referenced in the previous sentence, i.e., the term ``remains" is misleading.
- I did not quite understand the short paragraph on crowd-sourcing. Why is this even relevant?
 I believe it would suffice to simply state that cleansing is expensive...}

We appreciate the thorough feedback and have tightened up the writing in the introduction. In particular, we have consolidated the motivation to a single paragraph describing the expense of data cleaning. We include a single sentence referencing related work on crowdsourcing/human-guided data cleaning.


\subsection*{Review 2 Details}

\noindent\textbf{R2.1: The definition of ``clean data" is imprecise and not clear. It appears that ``cleaning" in this system refers to entity resolution, cleaning w.r.t. dependencies, and possibly other actions as needed by the application. This makes it difficult to gauge overall accuracy when there are different interpretations of cleanliness. It is not clear how entity resolution and cleaning w.r.t. dependencies can be done holistically.}

We added Section \ref{dmodel} to the paper which clarifies the supported data cleaning operations:

\emph{\sys supports data cleaning operations that can be represented as record-by-record transformations.
Formally, there exists a function (implemented via human or algorithm) that given a dirty record will return a unique clean record.
This does not cover errors that simultaneously affect multiple records such as record duplication or schema transformation problems.
We represent this operation as $C(\cdot)$ which can be applied to a record $r$ to recover the clean record $r' = C(r)$.
Therefore, for every $r \in R$ there exists a unique $r' \in R^*$, where $R^*$ is the hypothetical fully cleaned relation.
We assume that there is a featurizer $F(\cdot)$ that maps a record to a feature vector $x$ and a label vector $y$.
So each record corresponds to one training example for the downstream model.}

\vspace{0.5em}

Our appendix (Section \ref{set-of-r}) also describes an extension to this model where sets of records can be cleaned at once (e.g., a find-and-replace operation).

\vspace{0.5em}

\noindent\textbf{R2.2: The paper describes a problem setting focused on modelling the iterative cleaning process rather than actual data management problems. The paper may be better suited at an ML venue.}

Over the last 5 years a number of new results have expanded the scope of progressive and interactive data cleaning~\cite{mayfield2010eracer, DBLP:journals/pvldb/YakoutENOI11, yakout2013don, altowim2014progressive, whang2014incremental, papenbrock2015progressive, gruenheid2014incremental}.
However,  it turns out that the straight-forward integration of existing progressive data cleaning methods with model training can lead to error-prone and misleading results (Section \ref{correctness}).
Recognizing that data analytics is increasingly moving towards predictive modeling, \sys presents an initial exploration of this problem.  

\vspace{0.5em}

\noindent\textbf{R2.3: Missing references to related work on interactive data cleaning. For the comparative evaluation, 2/3 techniques are ML based techniques, not interactive data cleaning systems. See D2.\\
D2: Data cleaning systems have also considered interactive engagement with the user and the application of ML techniques. 
i) Mohamed Yakout, Laure Berti-Equille, Ahmed K. Elmagarmid. Don't be SCAREd: use SCalable Automatic REpairing with maximal likelihood and bounded changes. SIGMOD Conference 2013: 553-564
ii) Mohamed Yakout, Ahmed K. Elmagarmid, Jennifer Neville, Mourad Ouzzani, Ihab F. Ilyas.
Guided data repair. PVLDB 4(5): 279-289 (2011).
}

%\sys explores a different problem than the referenced related work.
The referenced works apply machine learning to improve the efficiency and/or reliability of data cleaning.
In contrast, we address the problem of corrupted training data affecting user-specified predictive models. 
The natural question is if the frameworks proposed in prior work can apply to this new problem setting, and we added Section \ref{alrw} to answer this:

\emph{Machine learning can be used to improve the efficiency and/or reliability of data cleaning~\cite{yakout2013don,gokhale2014corleone}.
For example, Yakout et al. train a model that evaluates the likelihood of a proposed replacement value \cite{yakout2013don}.
Another application of machine learning is value imputation, where a missing value is predicted based on those records without missing values.
Machine learning is also increasingly applied to make automated repairs more reliable with human validation \cite{DBLP:journals/pvldb/YakoutENOI11}.
Human input is often expensive and impractical to apply to entire large datasets.
Machine learning can extrapolate rules from a small set of examples cleaned by a human (or humans) to uncleaned data \cite{gokhale2014corleone, DBLP:journals/pvldb/YakoutENOI11}.
This approach can be coupled with active learning \cite{DBLP:journals/pvldb/MozafariSFJM14} to learn an accurate model with the fewest possible number of examples.
Intuitively, this means that the system queries a human only when the model indicates uncertainty.\\
In spirit, \sys is similar to active learning as both techniques seek to improve the efficiency of expensive manual or crowdsourced data cleaning by carefully selecting samples.
The natural question is whether existing active learning approaches can be applied to the problem of estimating models downstream from the data cleaning.
There are two key challenges that limit the applicability of existing frameworks: (1) correctness, and (2) efficiency. 
For (1), existing approaches are designed for training on homogeneous data, i.e., that are previously cleaned or known to be clean.
However, training on a mixture of clean data and yet-to-be cleaned dirty data can result in severe inaccuracies. 
One of the primary contributions of this work is an incremental model update algorithm with correctness guarantees for mixtures of data.
Even if the correctness problem is addressed, the downstream model problem leads to several new opportunities for optimizations (challenge (2)), which empirically improve on typical active learning criteria by up to an order of magnitude (Section \ref{eval}).}


\vspace{0.5em}

\textbf{R2.4: Sampling is an important part of the framework and influences the accuracy of the cleaning. Yet, there is little discussion on sampling rate, or how a sample is chosen.}

We have added a new section (Section 6), which is dedicated to describing the basic sampling algorithm.
Section 7 has been revised to describe optimizations to this algorithm.

\vspace{0.5em}

\textbf{R2.5: An end-to-end running example in Section 5 is needed to highlight the intuition of the cleaning process.}

We addressed this point with a number of examples (see response \textbf{M4}).


\vspace{0.5em}


\subsection*{Review 3 Details}
\noindent\textbf{R3.1: The authors do not distinguish between the system architecture and the individual issues that they are presenting.}

Response \textbf{M3} describes several revisions to the architecture including: separating problem formalization and architecture, discussing the data flow rather than the algorithms, and differentiating the essential components for correctness from optimizations.

\vspace{0.5em}

\noindent\textbf{R3.2: The paper uses lots of definitions, and a multitude of that do not necessarily contribute to readability.
Without being an expert in the field, I found it extremely difficult to follow the paper as it touches upon multiple problems at the same time: data cleaning, model training, convex analytics, etc., uses definitions, notation and lots of examples that did not allow me to have a global understanding of the work.\\
I would prefer to have a more focused paper on one of these aspects that has concrete goals and then, having an overview of the architecture of the system as a small section. I believe that the architecture should not be the focus and the skeleton of this paper. Instead, I believe that the authors could focus on the individual problems.}

We have discussed a number of specific textual revisions in response \textbf{M2} and \textbf{M3}. Here are a list of other revisions to improve the readability:

\begin{enumerate}
\item We have revised the entire paper to be more accessible and readable.

\item We have expanded the background section to provide a more detailed setup and context to the problem.
\item We revised the technical sections to first present a minimum viable solution that addresses the two subproblems (Section \ref{model-update} and Section  \ref{dist-samp}).
\item The next section (Section \ref{opti}) describes optional optimizations that can be applied in a number of practical cases.
\item Detailed derivations are now in the appendix, and the additional space has been used for three new examples in the technical sections (Sections \ref{model-update}-\ref{opti}).
\end{enumerate}
