{\noindent \normalsize \bf Dear SIGMOD Chair and Referees: }

\vspace{.5em}

We thank the reviewers and chair for the very helpful feedback on our paper. We have addressed all of the listed concerns and include references to the revised text in the cover letter. 
We have significantly revised and clarified the presentation and exposition
of the paper, and the major changes are summarized below:
\begin{enumerate}
\item We have revised the introduction to emphasize while a number of recent works apply Machine Learning internally to more efficiently or accurately repair dirty data, in this work, we explore the problem of data cleaning (learning-based or otherwise) prior to Machine Learning data analytics.
\item We acknowledge that in our initial submission we conflated the system architecture and the algorithmic implementation. To address this point, we formalized the two problems addressed in our paper: (1) the methodological problem of reliable analysis with progressive data cleaning and predictive modeling, and (2) the algorithmic problem of using information from the downstream model to prioritize data cleaning. 
\item Consequently, we revised the technical sections of the paper to describe our algorithmic solutions to these problems. 
\item We have expanded our running example that intuitively clarfies many of the key points and revised the presentation of the technical sections moving many of the unneccessary formalism and details to the appendix.
\item We included references to the relevant related work suggested by the reviewing committee.
\end{enumerate}
In the rest of the cover letter, we elaborate on these revisions in detail.

\vspace{0.5em}

\subsection*{Meta Review Details} 

We would first like to clarify an important point about the subject and scope of our paper in relation to related work.

There are a number of proposals for integrating data cleaning and machine learning \cite{gokhale2014corleone, DBLP:journals/pvldb/YakoutENOI11, yakout2013don}.
These works study the problem of using machine learning to clean the data.
In this problem setting, models are trained on the dirty data to predict the clean value from the dirty data given some supervisor feedback such as labels or examples.
Active Learning or statistical criteria, such as Value of Information, can be used to collect supervisor feedback that is most valuable to the cleaning model.
Once cleaned (or partially cleaned) the data then are ready for analytics.

In contrast, our work studies the problem of data cleaning when the subsequent analytics uses Machine Learning.
We show that when data are partially cleaned (i.e. a mix of dirty and cleaned data), models trained on this data can be arbitrarily incorrect.
This poses a methodological problem for data analysts who have to apply data cleaning before some statistical analysis.
Existing literature does not address the statistical validity of the subsequent analytics.
As a result, this problem is different from the existing literature and requires simultaneous data cleaning and incremental optimization.

However, ignoring the methodological problem, there is an aspect of our work that does build on the existing literature of data cleaning and active learning.
We explore how knowledge of the downstream statistical model can inform which records to clean.
This process requires a more general Active Learning framework than ones proposed related work.
For example, in Gokhale et al. \cite{gokhale2014corleone}, the authors use a technique called uncertainty sampling with random forests for entity resolution.
As one of the reviewers pointed out, active learning and progressive data cleaning has been widely applied entity resolution \cite{whang2014incremental, papenbrock2015progressive, gruenheid2014incremental}.
In almost all of the these works, the active learning technique only applies to learning problems with missing labels (i.e., are two records duplicates) not general data updates.
The work by Yakout et al. \cite{DBLP:journals/pvldb/YakoutENOI11, yakout2013don} explores a broader set of problems that can be prioritized using statistical criteria.
This line of work uses a rule-based model for data cleaning that offers richer semantics than our data cleaning model (record-by-record transformations).
Since the system knows the data consistency rules in advance, it can avoid the estimation step 
that we need to estimate how valuable a fix is w.r.t to the clean data. 

\vspace{0.5em}

\noindent\noindent \textbf{M1. There should be a formal vocabulary introduced early on. The exact idea of ``dirty" here can be hard to follow: what is the exact error type(s) that the system is intended to clean?}

\vspace{0.5em}

We thank the reviewers for this important feedback and clarified that our system applies to data cleaning operations that can be modeled as record-by-record mappings between two relations with the same schema.
This model is sufficiently expressive for our experiments and a number of real-world dirty data scenarios such as attribute value canonicalization and missing value filling.
We hope to explore additional data cleaning models that include record deduplication or schema mapping in future work.
We added the following clarification to Section \ref{dmodel}:

\emph{This paper explores a model for data error where attributes of records in a relation $R$ are incorrect or missing.
Formally, there are two relations $R_{dirty}$ and $R_{clean}$ with the same schema and a one-to-one mapping between rows.
Therefore, for every $r \in R_{dirty}$ there exists an $r' \in R_{clean}$, and a cleaning function $C(\cdot)$, is a function that maps each $r$ to its corresponding $r'$.
We assume that there is a featurization $F(\cdot)$ which is defined over both $R_{dirty}$ and $R_{clean}$ and maps rows to vectors in $\mathbb{R}^d$.
So each row corresponds to one training example in the predictive model.\\
Consider the data corruption in our motivating example where company names were inconsistently entered: ``Pfizer Inc.", ``Pfizer Incorporated", ``Pfizer".
Fixing this error can be represented as a record-by-record mapping.
The data cleaning function would pick one canonical representation for the company (e.g. ``Pfizer Inc.") and map all records with other values that refer to the same real-world entity to the canonical representation.
However, there are types of data cleaning that do not satisfy this model such as schema mapping, record deduplication, and data extractions that create additional columns.}

\vspace{0.5em}

\noindent\textbf{M2. Sections 5-7 are the technical core of the paper, and appear formal at the expense of aiding understanding. They appear to implement something that resembles active learning or bootstrapping, except inside the gradient descent loop. The motivation of some of this is not clear; is it necessary to integrate with the gradient descent? This is not how most active learning methods are implemented. Is it possible to implement these approaches in a way that is orthogonal to the SGD algorithm? The current writeup entangles some of these design choices.} 

We have clarified that \sys addresses two high-level problems that are decoupled from our solutions:

\emph{(1) the methodological problem of reliable analysis with progressive data cleaning and predictive modeling, and (2) the algorithmic problem of using information from the model to prioritize data cleaning} 

\vspace{0.5em}

We re-organized the paper to first describe the solution to problem (1). We argue that without an incremental optimization approach, there is a methodological problem of model training after progressive data cleaning. Figure \ref{update-arch1} describes straight-forward applications of progressive data cleaning and model training and why these approaches have issues.
These issues can occur whether or not the data cleaning is interactive or passive, as the problem is not with the data cleaning but with the model updating procedure.
The basic problem is that given the new data the model has to be incrementally updated rather than training on a mix of dirty and clean data.
The integration with SGD is one approach to addressing problem (1), and actually any incremental optimization technique can apply such as proximal methods and primal-dual methods.
The subsequent sections describe an algorithmic solution to problem (2). These approaches are orthogonal to the model updating problem and can apply with any incremental update framework. However, we motivate our specific choice of prioritization by describing optimality with respect to the SGD update.
In our revision, we have abstracted these two separate problems and describe them independently.

We have significantly revised the system architecture section to reflect these changes as well.
Section \ref{uinp} describes the inputs to the problem, Section \ref{updp} describes the update problem, and Section \ref{optp} describes the optimization problem.
The architecture section concludes by summarizing the componenents of our implementation of solutions to these problems and an end-to-end intuitive running example.

\vspace{0.5em}

\noindent\textbf{M3. In general, the distinction between an ``architecture" and an ``algorithm that fits into the architecture" is quite unclear. The problem with SGD/Active Learning above is one example.}

As mentioned in the previous response, we have addressed this problem by separating the methodological problem and the algorithmic problem. We have significantly revised the architecture described in Section 3 to make this clear:

\emph{\sys addresses two challenges in model training after progressive cleaning: (1) the methodological problem of reliable analysis or the \emph{Update Problem}, and (2) the algorithmic problem of using information from the model to prioritize data cleaning which we call the \emph{Optimization Problem}.
The \sys architecture in Figure \ref{sys-arch} describes an end-to-end system implementing our solutions to these two challenges.}

\vspace{0.5em}

Section \ref{updp} describes the update problem:

\emph{In the model update problem, \sys applies an update to the dirty model $\theta^{(d)}$, based on some newly cleaned data $S_{clean}$, to return $\theta^{new}$.}

\vspace{0.5em}

We clarify the requirements of the algorithm that addresses this problem in Section \ref{uinp} as well:

\emph{We call the update algorithm ``reliable" if the expected error is upper bounded by a monotonically decreasing function $f$ of the amount of cleaned data:
\[
\mathbb{E}(error(\theta^{new})) = O(f(\mid S_{clean} \mid))
\]
and is also upper bounded by a monotonically increasing function of the initial error:
\[
\mathbb{E}(error(\theta^{new})) = O(g(\| \theta^{(d)} - \theta^{(c)} \|))
\]
Intuitively, more cleaning and less initial error should imply more accuracy.
}

\vspace{0.5em}

Section \ref{optp} describes the optimization problem:

\emph{\sys optimizes the execution of the updates by carefully selecting $S_{clean}$.
The optimization problem is to select $S_{clean}$ such that the expected error $\mathbb{E}(error(\theta^{new}))$ is minimized.
\sys uses previously cleaned data to estimate the value of data cleaning on new records.
Then it draws a sample of records $S_{dirty} \subseteq R_{dirty}$. This is a non-uniform sample where each record $r$ has a sampling probability $p(r)$ based on the estimates.
We derive the optimal sampling distribution for the SGD updates, and show how the theoretical optimal can be approximated.
}

\vspace{0.5em}

The \sys framework is independent of the solutions to the update problem and the optimization problem.
However, in this work, our solutions to these problems inform each other as we use the theoretical analysis of SGD to inform how to guide sampling. 

\vspace{0.5em}

\noindent\textbf{M4. The paper, and especially the technical sections, would benefit enormously from a detailed running example showing how the algorithm works}

We have added a number of examples at the end of each of the technical sections. Section 3 (Architecture) ends with a intuitive end-to-end running example without technical details (Example \ref{archex}).
Section 4 (Update Problem) ends with an SVM example of how updates are propagated and calculated (Example \ref{upex}).
Section 5 (Detection) contains two examples for how the two different types of detectors can be used (Example \ref{detex1} and \ref{detex2}).
The optimization section ends with an example summarizing the mathematical concepts presented in that section.

\vspace{0.5em}

\noindent\textbf{M5.Some connections to related work that combines machine learning and data cleaning should be made. See the other reviewers' comments.}

We thank the reviewers for suggesting highly relevant related work, and now highlight these important results in the introduction:

\emph{When data cleaning is expensive, it is desirable to apply it \textbf{progressively}, where analysts can inspect early results with only $k \ll N$ records cleaned.
Progressive data cleaning is a well studied problem especially in the contex of entity resolution \cite{whang2014incremental, papenbrock2015progressive, gruenheid2014incremental}.
Increasingly, Active Learning \cite{settles2010active} or other statistical methods are applied to select records or contraint violations to clean in a way that maximizes the information gained \cite{DBLP:journals/pvldb/YakoutENOI11, gokhale2014corleone, yakout2013don}.\\
However, these frameworks do not consider the requirements of the downstream applications which can give valuable information on when and how to apply data cleaning.
While using downstream query semantics has been explored in the context of conjuctive queries \cite{DBLP:conf/sigmod/BergmanMNT15} and SQL aggregates \cite{wang1999sample}, it is important to recognize the growing popularity of predictive models in data analytics \cite{bdas, alexandrov2014stratosphere, crotty2014tupleware, hellerstein2012madlib}.
Predictive models rely on learning relationships between features and labels, and systematic data corruption \cite{taylor1982introduction} can mask or even introduce spurious new relationships.
Furthermore, the high dimensionality of these models can amplify small problems \cite{xiaofeature} resulting in error-prone predictions even when trained on mostly clean data.}

\vspace{0.5em}

We also added a detailed clarification in Section 2.2:

\vspace{0.5em}

\emph{There are also a number of recent approaches that apply Active Learning to reduce number of records that need to be cleaned by a human.
For example, Gokhale et al. \cite{gokhale2014corleone} uses Active Learning to select the most valuable pairwise comparisions to identify duplicates.
Yakout et al. \cite{yakout2013don,DBLP:journals/pvldb/YakoutENOI11} uses information theoretic and statistical criteria to prioritize data repair.
The key distinction is that in these problems Machine Learning is a part of the data cleaning technique.
Superivised machine learning methods are used to repair the data from training examples.
In our problem, the data cleaning, whether learning-based or otherwise, is external and at the end of the analysis pipeline.
In other words, we explore the problem of model training \emph{after} data cleaning as opposed to using a model \emph{for} data cleaning.\\
Our first goal is to find an update methodology that avoids Simpson's Paradox and the strong dependence on sample size.
Instead of mixing dirty and clean data, \sys uses a model trained on the dirty data as an initialization, and then iteratively updates this model using samples of clean data.
The intuition is that this algorithm smoothly transitions the model from one population (the dirty data) to another (the clean data), leading to provable guarantees about intermediate results.
Our next goal is to priotize which data to clean based on information from the model.
We design \sys to make greater progress at these small sample sizes using the dirty model as an initialization.
Doing so is not trivial since data may look unimportant to a dirty model but when cleaned are very important.
Also, data cleaning and model training can happen at very different time scales, we have to carefully budget our effort to ensure that any optimizations actually address rate-determining steps in the workflow.}

\subsection*{Review 1 Details} 

\noindent\textbf{R1.1: At first, the problem seems a bit too specialized. The abstract is too loaded with technical terms and a turn-off. This is then mitigated in the introduction. \\
As mentioned above, the abstract is (to me) overly technical and did not make me curious. I did not know off the bat what a convex loss model is, what importance sampling is, etc.}

We revised the abstract to be more accessible:

\emph{A perennial challenge in data analytics is presence of dirty data in the form of missing, duplicate, incorrect, or inconsistent values.
Although errors can be mitigated through data cleaning, it is often very time consuming.
One approach to making data cleaning more efficient is to leverage knowledge of the downstream data analysis to prioritize those records that are most likely to affect the result.
However, data analytics increasingly consist of statistical analysis and predictive modeling which can be highly sensitive to dirty data which is only partially cleaned.
This paper explores reliable model update procedures and priotization for progressive data cleaning.
We focus on two problems for a popular class of models called convex loss models (e.g., linear regression and SVMs): (1) the methodological problem of updating a model with partially clean data, and (2) the algorithmic problem of using information from the model to prioritize data cleaning.
The key insight of our framework is that data cleaning can be applied simultaneously with incremental optimization allowing for progressive cleaning while preserving provable properties.
Evaluation on four real-world datasets suggests that for a fixed cleaning budget, \sys returns more accurate models than uniform sampling and Active Learning when systematic corruption is sparse. }

\vspace{0.5em}

\noindent\textbf{R1.2: Poor embrace of the duplicate detection problem (see details below). Your model of the cleaner seems to preclude any duplicate detection, which certainly cannot happen on individual records. Also you extension for a set of record does not fit the problem of duplicate detection. This is in contrast, for instance, to your ER example in the second column of that page. Appendix A.1 is misleading here, as you mention with Example 7 ``in entity resolution problems..." but do not actually address that problem in the example. Fixing some common inconsistency is not entity resolution.}

We apologize for the confusing terminology and have revised our paper to clarify that we do not address record-level deduplication.
We formalized our data cleaning model to clarify this.
However, we intended to bring attention to the fact that some types of attribute level inconsistencies are addressed in similar ways to record deduplication.
For example, in our experimental dataset, the inconsistencies ``Pfizer Inc.", ``Pfizer Incorporated", and ``Pfizer" can be addressed using a blocking and matching procedure similar to that used in record deduplication but over a projection.
That said, we have removed references to entity resolution and described this in more precise terms.

\vspace{0.5em}

\noindent\textbf{R1.3: Cheated by using a narrower font than required. Will have trouble with camera ready copy if publisher insists on proper font.\\
- I would not use ``overview" as a verb...
- 3.2: the detector select -> the detector selects
- 4.3: Wrong quotation marks around ``learning".
- QED symbols on page 8 are ugly when placed directly after formula. 
- References need a clean up. Just as an example: Venue is missing for [24], year is mentioned 3 times for [8], [11], etc. Page numbers appear sporadically.}

We have addressed all of the formatting and copy editting issues.

\vspace{0.5em}

\noindent\textbf{R1.4:There is some related work specifically addressing progressive/incremental entity resolution. You might want to point your readers to this.
\\E.g.
\\- Incremental entity resolution on rules and data, Whang et al. VLDB Journal 2014
\\- Progressive duplicate detection, Papenbrock et al., TKDE 2015
\\- Incremental record linkage, Gruenheid et al., PVLDB 2014
\\- Another work that is related is ``Estimating the Number and Sizes of Fuzzy-Duplicate Clusters" by Heise et al. CIKM 2014, which also incrementally cleans samples of data to predict in this case the number of record matches.}

We have included the following references in the introduction:

\emph{Progressive data cleaning is a well studied problem especially in the contex of entity resolution \cite{whang2014incremental, papenbrock2015progressive, gruenheid2014incremental}.
Increasingly, Active Learning \cite{settles2010active} or other statistical methods are applied to select records or contraint violations to clean in a way that maximizes the information gained \cite{DBLP:journals/pvldb/YakoutENOI11, gokhale2014corleone, yakout2013don}.}

\vspace{0.5em}

\noindent\textbf{- Page 1, last paragraph in column 1 reads as if reference to [3] is a reaction to the work referenced in the previous sentence, i.e., the term ``remains" is misleading.
- I did not quite understand the short paragraph on crowd-sourcing. Why is this even relevant?
 I believe it would suffice to simply state that cleansing is expensive...}

We appreciate the thorough feedback and have tightened up the writing in the introduction. In particular, we have consolidated the motivation to a single paragraph describing the expense of data cleaning.


\subsection*{Review 2 Details}

\noindent\textbf{R2.1: The definition of ``clean data" is imprecise and not clear. It appears that ``cleaning" in this system refers to entity resolution, cleaning w.r.t. dependencies, and possibly other actions as needed by the application. This makes it difficult to gauge overall accuracy when there are different interpretations of cleanliness. It is not clear how entity resolution and cleaning w.r.t. dependencies can be done holistically.}

As mentioned above, we have included a formalizing of the data cleaning models addressed in this paper in Section 2.4. 

\vspace{0.5em}

\noindent\textbf{R2.2: The paper describes a problem setting focused on modelling the iterative cleaning process rather than actual data management problems. The paper may be better suited at an ML venue.}

We have revised the paper to emphasize its relevance to a database audience. We argue that the straight-forward application of existing progressive data cleaning methods, which have been widely proposed in the communitym can lead to error-prone and misleading results if used in conjuction with predictive modeling. This methodological concern is of interest to data management since variations of this process since our solution builds on ideas from online aggregation, incremental maintenance, and incremental analytics.

This is now reflected in our introduction:

\emph{When applied before predictive modeling, straight-forward applications of progressive data cleaning pose several methodological problems.
Suppose $k$ records are cleaned, but all of the remaining dirty records are retained in the dataset.
Training a model on a mixture of dirty and clean data can lead to misleading relationships in even simple scenarios (Figure \ref{update-arch1}).
An alternative is to clean $k$ records and to disregard all of the remaining dirty records (e.g., sampling \cite{wang1999sample}).
While this avoids the mixing problem, accurate model training may require a large amount of training data and $k$ examples may not be enough for a viable model.
Finally, both problems are compounded by sparsity, where if corrupted records are uncommon, a random subset of $k$ records may have relatively few examples of corruptions.
The errors and inefficiencies introduced by these three problems may dominate any gains from data cleaning, leading to unreliable or misleading conclusions about data or model quality.}

\vspace{0.5em}

\noindent\textbf{R2.3: Missing references to related work on interactive data cleaning. For the comparative evaluation, 2/3 techniques are ML based techniques, not interactive data cleaning systems. See D2.\\
D2: Data cleaning systems have also considered interactive engagement with the user and the application of ML techniques. 
i) Mohamed Yakout, Laure Berti-Equille, Ahmed K. Elmagarmid. Don't be SCAREd: use SCalable Automatic REpairing with maximal likelihood and bounded changes. SIGMOD Conference 2013: 553-564
ii) Mohamed Yakout, Ahmed K. Elmagarmid, Jennifer Neville, Mourad Ouzzani, Ihab F. Ilyas.
Guided data repair. PVLDB 4(5): 279-289 (2011).
}

We have included these reference both in related work and in Section 2.3:

\emph{There are also a number of recent approaches that apply Active Learning to reduce number of records that need to be cleaned by a human.
For example, Gokhale et al. \cite{gokhale2014corleone} uses Active Learning to select the most valuable pairwise comparisions to identify duplicates.
Yakout et al. \cite{yakout2013don,DBLP:journals/pvldb/YakoutENOI11} uses information theoretic and statistical criteria to prioritize data repair.
The key distinction is that in these problems Machine Learning is a part of the data cleaning technique.
Superivised machine learning methods are used to repair the data from training examples.
In our problem, the data cleaning, whether learning-based or otherwise, is external and at the end of the analysis pipeline.
In other words, we explore the problem of model training \emph{after} data cleaning as opposed to using a model \emph{for} data cleaning.
}

\vspace{0.5em}

\textbf{R2.4: Sampling is an important part of the framework and influences the accuracy of the cleaning. Yet, there is little discussion on sampling rate, or how a sample is chosen.}

We revised Section 6 to be more precise about the sampling which start with the following preface:

\emph{The model update received a sample with probabilities $p(\cdot)$.
Similar to Active Learning, we propose a sampling algorithm that selects the most valuable records to clean with higher probability.
We derive an approximation for the the sampling distribution that achieves the minimum variance update estimates. }

\vspace{0.5em}

\textbf{R2.5: An end-to-end running example in Section 5 is needed to highlight the intuition of the cleaning process.}

As mentioned above we have added a number of examples to describe the intuition for each of the technical sections.

\vspace{0.5em}


\subsection*{Review 3 Details}
\noindent\textbf{R3.1: The authors do not distinguish between the system architecture and the individual issues that they are presenting.}

As mentioned earlier we now distinguish between the methodological problem and the optimization problem.
We have revised to emphasize that \sys is one implementation of algorithms that address this problem.

\vspace{0.5em}

\noindent\textbf{R3.2: The paper uses lots of definitions, and a multitude of that do not necessarily contribute to readability.
Without being an expert in the field, I found it extremely difficult to follow the paper as it touches upon multiple problems at the same time: data cleaning, model training, convex analytics, etc., uses definitions, notation and lots of examples that did not allow me to have a global understanding of the work.\\
I would prefer to have a more focused paper on one of these aspects that has concrete goals and then, having an overview of the architecture of the system as a small section. I believe that the architecture should not be the focus and the skeleton of this paper. Instead, I believe that the authors could focus on the individual problems.}

We have revised the paper to improve readability. In particular, we have consolidated a number of sections and moved non-essential equations to the appendix. We do however believe that data cleaning needs to be understood as an essential part of the model training process. Data analysis is increasingly using predictive models and these models are highly sensitive to data cleaning procedures.
