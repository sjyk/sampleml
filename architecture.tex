\section{System Architecture}\label{arch}
In this section, we describe the \sys architecture and the basic algorithmic framework.
We will address the individual components in the subsequent sections.

\subsection{Overview}
Figure \ref{sys-arch} in the introduction overviews the entire framework.
The first step of \sys is \emph{initialization}.
In this step, there is a dirty relation $R$, a data cleaning technique $C(\cdot)$, and a dirty model $\theta^{(d)}$ trained on the dirty dataset. 
Optionally, \sys integrates with dirty data detection rules $D(\cdot)$ which selects the set of likely corrupted records from $R$.
If one is not provided, \sys starts by treating all of the data as dirty and tries to learn a detector as data are cleaned.
At initialization, there are two hyperparameters to set, the cleaning budget $k$ and the batch size $b$ (the number of iterations is $T = \frac{k}{b}$).
We discuss how to set $b$ and the tradeoffs in setting a larger or smaller $b$ in Section \ref{model-update}.

After initialization, \sys begins the cleaning and model update iterations.
The \emph{sampler} selects a sample of dirty data based on the batch size.
At this step, \sys can use the detector $D$ to narrow the sample to select only dirty data.
Once a sample is selected, the \emph{cleaner} applies $C(\cdot)$ to the dirty sample.
Then, after the sample is cleaned, the dirty model is updated by the \emph{updater}.

The next two steps in the architecture are feedback steps where the sampling distribution is updated for the next iteration.
The \emph{estimator} uses previously cleaned data to estimate the value of data cleaning on new records.
This information is used to guide sampling towards more valuable records.
After estimation, the detector $D(\cdot)$ is also updated based on cleaned data.
After all of the iterations are complete, the system returns the updated model.

To summarize the architecture in pseudocode:
\begin{enumerate}[leftmargin=1em]\scriptsize\sloppy
\item \texttt{Init(dirty\_data, cleaned\_data, dirty\_model, batch, iter)}
\item For each t in $\{1,...,T\}$
\begin{enumerate}
	\item \texttt{dirty\_sample $=$ Sampler(dirty\_data, sample\_prob, detector, batch)}
	\item \texttt{clean\_sample $=$ Cleaner(dirty\_sample)}
	\item \texttt{current\_model $=$ Updater(current\_model, sample\_prob, clean\_sample)}
	\item \texttt{cleaned\_data = cleaned\_data + clean\_sample}
	\item \texttt{dirty\_data = dirty\_data - clean\_sample}
	\item \texttt{sample\_prob $=$ Estimator(dirty\_data, cleaned\_data, detector)}
	\item \texttt{detector $=$ DetectorUpdater(detector, cleaned\_data)}
\end{enumerate}
\item \texttt{Output: current\_model}
\end{enumerate}

\subsection{Challenges and Formalization}
We highlight the important components and formalize the research questions that we explore in this paper. 

\vspace{0.5em}

\noindent\textbf{Detector (Section \ref{det}). } The first challenge in \sys is dirty data detection. In this step, we select a candidate set of dirty records $R_{dirty} \subseteq R$. We will discuss two techniques to do this: (1) an a priori case, and (2) and an adaptive case. In the a priori case, we know which data is dirty in advance. In the adaptive case, we train a classifier based on data that we have already cleaned to select the dirty data.

\vspace{0.5em}

\noindent\textbf{Sampler (Section \ref{dist-samp}). } We take a sample of the records $S_{dirty} \subseteq R_{dirty}$. This is a non-uniform sample where each record $r$ has a sampling probability $p_r$.
We will derive the optimal sampling distribution, and show how the theoretical optimal can be approximated by the next estimator.

\vspace{0.5em}

\noindent\textbf{Cleaner (User-Specified). } We take the sample of the records $S_{dirty}$, and apply the analyst-specified data cleaning $C(\cdot)$. In this paper, we focus on a record-by-record cleaning model where the function $C$ is applied to a record and produces the clean record:
\[
S_{clean} = \{C(r) : \forall r \in S_{dirty}\}
\]
This allows us to measure the performance of \sys in terms of model error as a function of sample size. The record-by-record cleaning model is not a fundemental restriction of our approach, and in our extensions (Section \ref{set-of-r}) , we discuss an also compatible ``set of records" cleaning model. Consider the case where an analyst finds a dirty record, and is able to fix all records (possibly outside the sample) with same error throughout the dataset efficiently.

\vspace{0.5em}

\noindent\textbf{Updater (Section \ref{model-update}). } We update the model $\theta^{(t)}$ based on the newly cleaned data $F(S_{clean})$ resulting in $\theta^{(t+1)}$. Analyzing the model update procedure as a stochastic gradient descent algorithm will help us derive the sampling distribution and estimation.

\vspace{0.5em}

\noindent\textbf{Estimator (Section \ref{sampling}): } The estimator approximates the optimal distribution derived in the Sample step. Based on the change between $F(S_{clean})$ and $F(S_{dirty})$, we direct the next iteration of sampling to select points that will have changes most valuable to the next model update.

\iffalse
\subsection{Optimizations}
There are three aspects of \sys, that allow us to achieve this design point: error partitioning, gradient-based model update (Section \ref{model-update}), estimate-driven sampling (Section \ref{sampling}).

\vspace{0.5em}

\noindent\textbf{Partitioning Dirty and Clean Data: } In many applications, enumerating the set of corrupted records is much easier than cleaning them. For example, we may be able to select the set of rows that have missing values but actually filling those missing values is expensive. Likewise, in the constraint literature, selecting a set of rows that have a violated constraint can be done in polynomial time, however, fixing the constraints is NP-Hard.
In our error detection step, we partition the dirty and clean data.
Partitioning serves two purposes: (1) it reduces the variance of our updates because we can cheaply scan over data we know that is clean, and (2) it increases the fraction of actually dirty records in the candidate batch.
A good example of why we need the second objective is seen in the context of crowdsourcing.
If we have a crowdworker clean records, we will have to pay them for the task whether or not the record required cleaning.
To efficiently use this partitioning, we need a database solution indexing dirty and clean data.

\vspace{0.5em}

\noindent\textbf{Gradient-Based Updates: } In \sys, we start with a dirty model and then make an update using a gradient step. Here, we can draw an analogy to Materialized View maintenance, since after all, a model parametrized by $\theta$ is just a table of floating point numbers.
Krishnan et al. proposed a technique called sample view cleaning, in which they take a clean sample of data and propagate the updates to a Materialized View.
Similarly, in this work, we take the information from a sample of cleaned data and propagate an update with the gradient.

\vspace{0.5em}

\noindent\textbf{Estimate-Driven Sampling: } Repair is the most expensive step in the workflow, so optimizing for scan cost may lead to negligible overall time improvements.
We can sacrifice a small overhead in pre-computation for each data point to determine its value to the model and select a sampling distribution accordingly.
Intuitively, while each iteration has an increased cost, it also makes more progress towards the optimum.
\fi


