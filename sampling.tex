\section{Sampling}\label{dist-samp}
In the previous section, we assumed that our model update recieved a sample with probabilities $p(r)$.
In this section, we derive an optimal sampling problem that directly follows from our analysis of our update rule via SGD.
It will turn out that the solution to the optimal sampling problem is not realizable in practice (as it depends on knowing the cleaned value), but we can use this to inform the next section where we estimate the cleaned value.

\subsection{Goals and Challenges}
In the Machine Learning and Optimization literature, SGD algorithms are optimized to avoid scanning the entire data.
Uniform sampling is cheap so it is the preferred solution.
However, data cleaning costs can be many orders of magnitude higher than model training.
As a result, uniform sampling may not be the most efficient option.
We can sacrifice computational overhead by precomputing some results over the entire data for savings during the data cleaning phase.
We formulate this problem as an optimal sampling problem where we want to compute the sampling probabilities $p(r)$ that maximize the accuracy of our updates.

\subsection{Optimal Sampling Problem}
Recall that the convergence rate of an SGD algorithm is bounded by $\sigma^2$ which is the variance of the gradient.


\begin{definition}[Optimal Sampling Problem]
Given a set of candidate dirty data $R_{dirty}$, $\forall r \in R_{dirty}$ find sampling probabilites $p(r)$ such that over all samples $S$ of size $k$ it minimizes:
\[
\mathbb{E}(\|g_S - g^*\|^2)
\]
\end{definition}
In other words, we want to most accurately (in the mean squared error sense) estimate the gradient.

To construct these sampling probabilities, we first need the following lemma:
\begin{lemma}\label{impsample}
Given a set of real numbers $A = \{a_1,...,a_n\}$, let $\hat{A}$ be 
a sample with replacement of $A$ of size k.
If $\mu$ is the mean $\hat{A}$, the sampling distribution that minimizes
 the variance of $\mu$, i.e., the expected square error, is $p(a_i) \propto a_i$.
\end{lemma}
\begin{proof}[Sketch]
This proof follows from \cite{mcbook}, as it is a straight forward importance sampling result.
It is easy to verify that for sampling probabilities $p(a_i)$, that the unbiased
estimate of the mean is:
\[
\mu = \frac{1}{nk}\cdot\sum_{i\in\hat{A}}^k \frac{a_i}{p_i}
\]
If we minimize the variance $Var(\mu)$, we can find that:
\[
p_i = \frac{\mid a_i \mid }{\sum_i \mid a_i \mid}
\]
\end{proof}

\iffalse
The variance of this estimate is given by:
\[
Var(\mu) = \mathbb{E}(\mu^2)-\mathbb{E}(\mu)^2
\] 
Since the estimate is unbiased, we can replace $\mathbb{E}(\mu)$ with the average of $A$:
\[
Var(\mu) = \mathbb{E}(\mu^2)-\bar{A}^2
\]
Since $\bar{A}$ is deterministic, we can remove that term during minimization.
Furthermore, we can write $\mathbb{E}(\mu^2)$ as:
\[
\mathbb{E}(\mu^2) = \frac{1}{n^2}\sum_i^n \frac{a_i^2}{p_i}
\]
Then, we can solve the following optimization problem (removing the proportionality of $\frac{1}{n^2}$) over the set of weights $P=\{p(a_i)\}$:
\[
\min_{P} \sum_i^N \frac{a_i^2}{p_i}
\]
\[
\text{subject to: } P > 0, \sum P = 1
\]
Applying Lagrange multipliers, an equivalent unconstrained optimization problem is:
\[
\min_{P > 0,\lambda > 0} \sum_i^N \frac{a_i^2}{p_i} + \lambda \cdot (\sum P - 1)
\]
If, we take the derivatives with respect to $p_i$ and set them equal to zero:
\[
-\frac{a_i^2}{2 \cdot p_i^2} + \lambda = 0
\]
If, we take the derivative with respect to $\lambda$ and set it equal to zero:
\[
\sum P - 1
\]
Solving the system of equations, we get:
\[
p_i = \frac{\mid a_i \mid }{\sum_i \mid a_i \mid}
\]
\fi

Based on Lemma \ref{impsample}, we now derive an optimal sampling distribution.
Lemma \ref{impsample} shows that when estimating a mean of numbers with sampling, the distribution with optimal variance is where we sample proportions to the values.
This is insight leads to a direct higher-dimensional generalization, where at iteration $t$ we should sample the records in $R_{dirty}$ with probabilities:
\[
p_i \propto \|\nabla\phi(x^{(clean)}_i,y^{(clean)}_i,\theta^t)\| \blacksquare
\]

However, in our case, it leads to a chicken-and-egg problem.
The optimal sampling distribution requires knowing $(x^{(clean)}_i,y^{(clean)}_i)$, however, we have to sample and clean those points to get those values.
In the next section, we discuss how to inexpensively approximate this optimal distribution.
As our technique can work with \emph{any} distribution, we are guaranteed convergence no matter how inaccurate this approximation.
However, a better approximation will lead to an improved convergence rate.
