\section{Efficiency With Sampling}\label{dist-samp}
The model update received a sample with probabilities $p(\cdot)$.
For any distribution where  $p(\cdot) > 0$, we can preserve correctness.
\sys uses a sampling algorithm that selects the most valuable records to clean with higher probability. 

\subsection{Oracle Sampling Problem}
Recall that the convergence rate of an SGD algorithm is bounded by $\sigma^2$ which is the variance of the gradient.
Intuitively, the variance measures how accurately the gradient is estimated from a uniform sample.
Other sampling distributions, while preserving the sample expected value, may have a lower variance.
Thus, the oracle sampling problem is defined as a search over sampling distributions to find the minimum variance sampling distribution.

\begin{definition}[Oracle Sampling Problem]
Given a set of candidate dirty data $R_{dirty}$, $\forall r \in R_{dirty}$ find sampling probabilities $p(r)$ such that over all samples $S$ of size $k$ it minimizes:
\[
\mathbb{E}(\|g_S - g^*\|^2)
\]
\end{definition}
It can be shown that the optimal distribution over records in $R_{dirty}$ is probabilities proportional to:
\[
p_i \propto \|\nabla\phi(x^{(c)}_i,y^{(c)}_i,\theta^{(t)})\|
\]
This is an established result, for thoroughness, we provide a proof in the appendix (Section \ref{impsample-deriv}), but intuitively, records with higher gradients should be sampled with higher probability as they affect the update more significantly.
However, this cannot exclude records with lower gradients as that would induce a bias hurting convergence.
The problem is that this optimal distribution leads to a chicken-and-egg problem:
the optimal sampling distribution requires knowing $(x^{(c)}_i,y^{(c)}_i)$, however, cleaning is required to know those values.

\subsection{Dirty Gradient Solution}
Such an oracle does not exist, and one solution is to use the gradient w.r.t to the dirty data:
\[
p_i \propto \|\nabla\phi(x^{(d)}_i,y^{(d)}_i,\theta^{(t)})\|
\]
It turns out that this solution works reasonably well in practice on our experimental datasets and has been studied in Machine Learning as the Expected Gradient Length heuristic \cite{settles2010active}.
The contribution in this work is integrating this heuristic with statistically valid updates.
However, intuitively, approximating the oracle as closely as possible can result in improved prioritization.
The subsequent section describes two components, the detector and estimator, that can be used to achieve this.
Our experiments suggest up-to a 2x improvement in convergence when using these optional optimizations (Section \ref{comp}).