\section{Detection}\label{det}
An important part of \sys is dirty data detection.
When we sample data to clean, we want to ensure that the data that we are sampling is actually dirty.
Therefore, before we can sample, we have to detect whether the record is dirty.

\subsection{Goals}
The detection component needs to give us two important pieces of information about a record.
First, it needs to tell us if a record is dirty or clean.
In this framework, we favor false positives (record is clean but marked as dirty) over false negatives as the false negatives will never get sampled and cleaned.
Next, the detection also needs to tell us what is wrong with the record.
This is important so we can estimate how valuable cleaning will be for this record.
There are two cases that we explore in \sys: \emph{a priori} and adaptive.
In the a priori case, we recognize that for many data cleaning methodologies, we can efficiently select the set of dirty records without repair.
In the adaptive case, we relax this assumption, and explore how we can learn which records are dirty and clean with a classifier.

\subsection{A Priori Case}
For many types of dirtiness, it is possible to efficiently enumerate a set of corrupted records and enumerate what is wrong with these records.

\begin{definition}[A Priori Detection]
Let $r$ be a record in $R$. An a priori detector is a detector is a detector that returns a Boolean of whether the record is dirty and a set of columns $e_r$ that are dirty.
\[
D(r) = (\{0,1\}, e_r)
\]
From the set of columns that are dirty, we can find the corresponding features that are dirty $f_r$.
\end{definition}

\noindent We highlight example use cases of this definition using data cleaning methodologies proposed in the liteature.

\vspace{0.5em}

\noindent\textbf{Constraint-based Repair: }
One model for handling errors in database declaring a set of constraints on a database and 
iteratively fixing records such that the constraints are satisfied \cite{DBLP:journals/pvldb/YakoutENOI11, DBLP:journals/pvldb/FanLMTY10, khayyat2015bigdansing}.

\vspace{0.5em}

\emph{Detection. } Let $\Sigma$ be a set of constraints on the relation $\mathcal{R}$. 
In the detection step, we select a subset of records $\mathcal{R}_{dirty} \subseteq \mathcal{R}$ that violate at least one constraint.
The set $e_r$ is the set of columns for each record which have a constraint violation. 

\begin{example}
Consider our EEG data running example, for an example of constraint-based cleaning.
We can add a constraint that one of the electrical signals cannot be greater than 4V.
For all records whose value is above 4V, we would select them in the error detection step.
\end{example}

\vspace{0.5em}

\noindent\textbf{Entity Resolution: }
Another common data cleaning task is Entity Resolution \cite{gokhale2014corleone, DBLP:journals/pvldb/KopckeTR10, wang2012crowder}.
A common pattern in Entity Resolution is to split up the operation into two steps: blocking and matching.
In blocking, attributes that should be the same are coarsely grouped together.
In matching, those coarse groups are resolved to a set of distinct entities.

\vspace{0.5em}

\emph{Detection. } This is the matching step. Let $S$ be a similarity function that takes two records and returns a value in $[0,1]$ (1 most similar and 0 least similar). For some threshold $t$, $S$ defines a similarity relationship between two records $r$ and $r'$:
\[
r \approx r' : S(r,r') \ge t
\] 
In the detection step, $R_{dirty}$ is the set of records that have at least one other record in the relation that satisfy $r \approx r'$.
The set $e_r$ is the attributes of $r$ that have entity resolution problems.

\begin{example}
An example of an Entity Resolution problem is where the patient's gender is inconsistently represented (e.g., ``Male", ``M", ``Man"). 
We can define a similarity relationship $(\text{first char =} 'M')$ and select all records that satisfy this condition.
\end{example}

\subsection{Adaptive Detection}
Next, we explore how we can handle the case where the detection is learned as we clean data.
The challenge in formulating this problem is that we not only need to know whether or not a record is dirty, but also how it is dirty (e.g. $e_r$ in the a priori case).
We try to formulate a technique that generalizes what we did before.
Instead of assuming that we know which features are corrupted, let us say that we know that there are $u$ classes of data corruption.
As the analyst cleans data, she tags dirty data with one of the $u$ classes.
Then, the detection problem reduces to a multiclass classification problem.
To address this problem, we can use any multiclass classifier, and we use an all-versus-one SVM in our experiments.

When an example $(x,y)$ is cleaned, the repair step also has to provide a label of which of the ${\text{clean}, 1,2,...,u}$ classes it belongs. It is possible that $u$ increases each iteration as more types of dirtiness are discovered. 
Thus, we after cleaning $k$ records, we have a dataset of $k$ records labeled with $u+1$ classes (including not dirty).

\begin{definition}[Adaptive Case]
To select $R_{dirty}$, we select the set of records for which $\kappa$ give a positive error classification (i.e., one of the $u$ error classes).
After each sample of data is cleaned, the classifier $\kappa$ is retrained.
So the result is:
\[D(r) = (\{1,0\},\{1,...,u+1\})\]
\end{definition}

As we mentioned before, we favor false positive errors over false negative errors when detecting dirty data.
Many types of classifiers allow users to tradeoff precision and recall.
In other words, we can also select any record within some level of confidence of the classification margin.
For an SVM, we may only classify a point as clean if it is sufficiently far from the margin.
Or for Logistic Regression, we may do so if its class likelihood is over 80\%.

\vspace{0.5em}

\noindent\textbf{Interactive Data Cleaning with OpenRefine.}
\begin{example}
Consider our analyst using an interative tool such as OpenRefine \cite{openrefine} to clean her data. 
She takes a sample of data from the entire dataset.
She then uses the tool to determine which records are corrupted and dirty.
As she marks more records as dirty, the system learns what features determines a dirty record.
\end{example}