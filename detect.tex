\section{Detection}\label{det}
To maximize the benefit of data cleaning, detection ensures that sampling draws records likely to be dirty.

\subsection{Goals}
The detector returns two important pieces of information about a record: (1) whether the record is dirty, and (2) if it is dirty, what is wrong with the record.
The sampler can use (1) to select a subset of dirty records to sample at each batch. 
The estimator can use (2) estimate the value of data cleaning based on other records with the same corruption.
\sys supports two types of detectors \emph{a priori} and \emph{adaptive}.
In the \emph{a priori} case, there is a way to select the set of dirty records before any cleaning.
In the adaptive case, detection is learned as data is cleaned.

\subsection{A Priori Case}
For many types of dirtiness such as missing attribute values and constraint violations, it is possible to efficiently enumerate a set of corrupted records and enumerate what is wrong with these records.

\begin{definition}[A Priori Detection]
Let $r$ be a record in $R$. An a priori detector is a detector that returns a Boolean of whether the record is dirty and a set of columns $e_r$ that are dirty.
\[
D(r) = (\{0,1\}, e_r)
\]
From the set of columns that are dirty, we can find the corresponding features that are dirty $f_r$ and labels that are dirty $l_r$.
\end{definition}

\noindent We highlight example use cases of this definition using data cleaning methodologies proposed in the liteature.

\vspace{0.5em}

\noindent\textbf{Constraint-based Repair: }
One model for detecting errors involves declaring constraints on the database.

\vspace{0.5em}

\emph{Detection. } Let $\Sigma$ be a set of constraints on the relation $\mathcal{R}$. 
In the detection step, the detector select a subset of records $\mathcal{R}_{dirty} \subseteq \mathcal{R}$ that violate at least one constraint.
The set $e_r$ is the set of columns for each record which have a constraint violation. 

\begin{example}
An example of a constraint on the running example dataset is the \texttt{status} of
a contribution can be only ``allowed" or ``disallowed".
Any other value for \texttt{status} is an error.
\end{example}

\vspace{0.5em}

\noindent\textbf{Entity Resolution: }
Another common data cleaning task is Entity Resolution \cite{gokhale2014corleone, DBLP:journals/pvldb/KopckeTR10, wang2012crowder}.
Entity Resolution is the problem of standardizing attributes that represent the same real world entity.
A common pattern in Entity Resolution is to split up the operation into two steps: blocking and matching.
In blocking, attributes that should be the same are coarsely grouped together.
In matching, those coarse groups are resolved to a set of distinct entities.

\vspace{0.5em}

\emph{Detection. } Detection for entity resolution problems is the matching step. Let $S$ be a similarity function that takes two records and returns a value in $[0,1]$ (1 most similar and 0 least similar). For some threshold $t$, $S$ defines a similarity relationship between two attributes $r(a)$ and $r'(a)$:
\[
r(a) \approx r'(a) : S(r(a),r'(a)) \ge t
\] 
In the detection step, $R_{dirty}$ is the set of records that have at least one other record in the relation that satisfies $r(a) \approx r(a)'$.
The set $e_r$ is the set of attributes of $r$ that have entity resolution problems.

\begin{example}
An example of an Entity Resolution problem is seen in our earlier example about corporation names e.g. ``Pfizer Inc.", ``Pfizer Incorporated", ``Pfizer".. 
We can define a similarity relationship $WeightedJaccard(r1,r2)>0.8$ and select all records that satisfy this condition (their Weighted Jaccard Similarity is greater than 0.8).
\end{example}

\subsection{Adaptive Detection}
\emph{A priori} detection is not possible in all cases.
The detector also supports adaptive detection where detection is learned from previously cleaned data.
Note that this ``learning" is distinct from the ``learning" at the end of the pipeline.
The challenge in formulating this problem is that detector needs to describe how the data is dirty (e.g. $e_r$ in the \emph{a priori} case).
The detector achieves this by taxonomizing the corruption into $u$ classes.
These classes are corruption categories that do not necessarily align with features, but every records is classified with at most one category.
For example, suppose there are records with outliers and missing values, there are three classes of corruption: outliers, missing values, and both.

When using adaptive detection, the repair step has to clean the data and report to which of the $u$ classes the corrupted record belongs.
When an example $(x,y)$ is cleaned, the repair step labels it with one of the ${\text{clean}, 1,2,...,u+1}$ classes (including one for ``not dirty").
It is possible that $u$ increases each iteration as more types of dirtiness are discovered. 
Then, the detection problem reduces to a multiclass classification problem.
This problem can be addressed by any multiclass classifier, and we use an all-versus-one SVM in our experiments.
Since this classifier is internal to our system, it does not have to be a convex model (i.e., it can be a Decision Tree or Random Forest).

\begin{definition}[Adaptive Case]
To select $R_{dirty}$, we select the set of records for which $\kappa$ gives a positive error classification (i.e., one of the $u$ error classes).
After each sample of data is cleaned, the classifier $\kappa$ is retrained.
So the result is:
\[D(r) = (\{1,0\},\{1,...,u+1\})\]
\end{definition}

\vspace{0.5em}

\noindent\textbf{Adaptive Detection With Open Refine: }
\begin{example}
OpenRefine is a spreadsheet-based tool that allows users to explore and transform data.
However, it is limited to clean data that can fit in memory on a single computer.
Since the cleaning operations are coupled with data exploration, we do not know what is dirty in advance (the analyst may discover new errors as she cleans).

Suppose our analyst wants to use OpenRefine to clean our running example dataset with \sys.
She takes a sample of data from the entire dataset and uses the tool to discover errors.
For example, she finds that some drugs are incorrectly classified as both drugs and devices.
She then clears the device attribute for all records that have the drug name in question.
Every time she makes a batch data transformation (i.e., cleaning the device attribute), we can list the set of records that have changed.
Each transformation becomes and error class, and the records that have changed records become positive training examples for a classifier to guide future samples.
\end{example}