\section{Detection}\label{det}
When we sample data to clean, we want to ensure that the data that we are sampling is actually dirty.
Therefore, the first step is to select a set of likely dirty records.

\subsection{Goals}
The detection component needs to give us two important pieces of information about a record: (1) whether the record is dirty, and (2) if it is dirty, what is wrong with the record.
From (1), we can select a subset of dirty records to sample at each batch. 
(2) is important so we can estimate how valuable cleaning will be for this record.
There are two cases that we explore in \sys: \emph{a priori} and \emph{adaptive}.
In the a priori case, we recognize that for many data cleaning methodologies, we can efficiently select the set of dirty records without repair.
In the adaptive case, we relax this assumption, and explore how we can learn which records are dirty and clean with a classifier.

\subsection{A Priori Case}
For many types of dirtiness such as missing attribute values and constraint violations, it is possible to efficiently enumerate a set of corrupted records and enumerate what is wrong with these records.

\begin{definition}[A Priori Detection]
Let $r$ be a record in $R$. An a priori detector is a detector that returns a Boolean of whether the record is dirty and a set of columns $e_r$ that are dirty.
\[
D(r) = (\{0,1\}, e_r)
\]
From the set of columns that are dirty, we can find the corresponding features that are dirty $f_r$ and labels that are dirty $l_r$.
\end{definition}

\noindent We highlight example use cases of this definition using data cleaning methodologies proposed in the liteature.

\vspace{0.5em}

\noindent\textbf{Constraint-based Repair: }
One model for handling errors in database declaring a set of constraints on a database.
Data are cleaned iteratively until the constraints are satisfied \cite{DBLP:journals/pvldb/YakoutENOI11, DBLP:journals/pvldb/FanLMTY10, khayyat2015bigdansing}.

\vspace{0.5em}

\emph{Detection. } Let $\Sigma$ be a set of constraints on the relation $\mathcal{R}$. 
In the detection step, we select a subset of records $\mathcal{R}_{dirty} \subseteq \mathcal{R}$ that violate at least one constraint.
The set $e_r$ is the set of columns for each record which have a constraint violation. 

\begin{example}
An example of a constraint for our running example dataset is that the \texttt{status} of
a contribution can be only ``covered" or ``non-covered".
Any other value for \texttt{status} is an error.
\end{example}

\vspace{0.5em}

\noindent\textbf{Entity Resolution: }
Another common data cleaning task is Entity Resolution \cite{gokhale2014corleone, DBLP:journals/pvldb/KopckeTR10, wang2012crowder}.
Entity Resolution is the problem of standardizing attributes that represent the same real world entity.
A common pattern in Entity Resolution is to split up the operation into two steps: blocking and matching.
In blocking, attributes that should be the same are coarsely grouped together.
In matching, those coarse groups are resolved to a set of distinct entities.

\vspace{0.5em}

\emph{Detection. } This is the matching step. Let $S$ be a similarity function that takes two records and returns a value in $[0,1]$ (1 most similar and 0 least similar). For some threshold $t$, $S$ defines a similarity relationship between two attributes $r(a)$ and $r'(a)$:
\[
r(a) \approx r'(a) : S(r(a),r'(a)) \ge t
\] 
In the detection step, $R_{dirty}$ is the set of records that have at least one other record in the relation that satisfy $r(a) \approx r(a)'$.
The set $e_r$ is the attributes of $r$ that have entity resolution problems.

\begin{example}
An example of an Entity Resolution problem is seen in our earlier example about corporation names e.g. ``Pfizer Inc.", ``Pfizer Incorporated", ``Pfizer".. 
We can define a similarity relationship $WeightedJaccard(r1,r2)>0.8$ and select all records that satisfy this condition (their Weighted Jaccard Similarity is greater than 0.8).
\end{example}

\subsection{Adaptive Detection}
Next, we explore how we can handle the case where the detection is not known in advance.
Based on what we have already cleaned, we can learn a detection classifier (note that this ``learning" is distinct from the ``learning" at the end of the pipeline).
The challenge in formulating this problem is that we not only need to know whether or not a record is dirty, but also how it is dirty (e.g. $e_r$ in the a priori case).
Instead of assuming that we know which features are corrupted, let us say that we know that there are $u$ classes of data corruption.
These classes are corruption categories that do not necessarily align with features, but every records is classified with at most one category.
For example, suppose we have outliers and missing values, there are three classes of corruption: outliers, missing values, and both.
As the analyst cleans data, she tags dirty data with one of the $u$ classes.
Then, the detection problem reduces to a multiclass classification problem.
To address this problem, we can use any multiclass classifier, and we use an all-versus-one SVM in our experiments.
Since this classifier is internal to our system, it does not have to be a convex model (i.e., it can be a Decision Tree or Random Forest).

When an example $(x,y)$ is cleaned, the repair step also has to provide a label to which of the ${\text{clean}, 1,2,...,u}$ classes it belongs. It is possible that $u$ increases each iteration as more types of dirtiness are discovered. 
Thus, after cleaning $k$ records, we have a dataset of $k$ records labeled with $u+1$ classes (including one for ``not dirty").

\begin{definition}[Adaptive Case]
To select $R_{dirty}$, we select the set of records for which $\kappa$ gives a positive error classification (i.e., one of the $u$ error classes).
After each sample of data is cleaned, the classifier $\kappa$ is retrained.
So the result is:
\[D(r) = (\{1,0\},\{1,...,u+1\})\]
\end{definition}

We highlight an example of adaptive detection using an interative data cleaning tool such as OpenRefine \cite{openrefine}.

\vspace{0.25em}

\noindent\textbf{Interactive Data Cleaning.}
\begin{example}
OpenRefine is a spreadsheet-based tool that allows users to explore and transform data.
However, it is limited to clean data that can fit in memory on a single personal computer.
Since the cleaning operations are coupled with data exploration, we do not know what is dirty in advance (the analyst may discover new errors as she cleans).

Suppose our analyst wants to use OpenRefine to clean our running example dataset with \sys.
She takes a sample of data from the entire dataset and uses the tool to discover errors.
For example, she finds that some drugs are incorrectly classified as both drugs and devices.
She then clears the device attribute for all records that have the drug name in question.
Every time she makes a batch data transformation (i.e., clearning the device attribute), we can list the set of records that have changed.
Each transformation becomes and error class, and the records that have changed records become positive training examples for a classifier to guide future samples.
\end{example}