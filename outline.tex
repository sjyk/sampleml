\documentclass{vldb}


\usepackage{enumitem}
\usepackage{framed}
\usepackage{cprotect}
\usepackage{enumitem}
\usepackage{listings}
\usepackage{amstext}
\usepackage{amstext}
\usepackage{pdfpages}
\usepackage{alltt}
\usepackage{epstopdf}
\usepackage{xspace,colortbl}
\usepackage[USenglish]{babel}
\usepackage{multirow}
\usepackage{url}
\usepackage{subfigure}
\usepackage{graphicx}%%
\usepackage{amssymb}
\usepackage{fmtcount}
\usepackage{amsfonts}
\usepackage{xspace}
\usepackage{amsmath}
\usepackage{multirow}
\usepackage[mathscr]{eucal}
%\usepackage{psfrag}
\usepackage{colortbl}

\usepackage{bm}
\usepackage{times}
\usepackage[nospace]{cite}
\usepackage{csquotes}

\lstset{basicstyle=\large,breaklines=true,language=SQL,belowcaptionskip=.1\baselineskip}

\linespread{0.963}

\makeatletter
\def\@copyrightspace{\relax}
\makeatother

\begin{document}

\setlength{\belowdisplayskip}{0.5pt} \setlength{\belowdisplayshortskip}{1pt}
\setlength{\abovedisplayskip}{0.5pt} \setlength{\abovedisplayshortskip}{1pt}
\setlength{\belowcaptionskip}{-10pt}
\selectfont

\newtheorem{theorem}{Theorem}
\newtheorem{example}{Example}
\newtheorem{definition}{Definition}
\newtheorem{problem}{Problem}
\newtheorem{property}{Property}
\newtheorem{proposition}{Proposition}
\newtheorem{lemma}{Lemma}
\newtheorem{corollary}{Corollary}

\newcommand{\cond}{\textrm{pred}\xspace}
\newcommand{\dataset}{data set\xspace}
\newcommand{\datasets}{data sets\xspace}
\newcommand{\spview}{\textsf{SPView}\xspace}
\newcommand{\fjview}{\textsf{FJView}\xspace}
\newcommand{\aggview}{\textsf{AggView}\xspace}
\newcommand{\hashfunc}[1]{\textsf{hash}(#1)\xspace}
\newcommand{\hashop}{\textsf{hash}\xspace}
\newcommand{\nsc}{\textsf{NormalizedSC}\xspace}
\newcommand{\rsc}{\textsf{RawSC}\xspace}

\newcommand{\avgfunc}{\ensuremath{\texttt{avg} }\xspace}
\newcommand{\maxfunc}{\ensuremath{\texttt{max} }\xspace}
\newcommand{\minfunc}{\ensuremath{\texttt{min} }\xspace}
\newcommand{\histfunc}{\ensuremath{\texttt{histogram\_numeric} }\xspace}
\newcommand{\countfunc}{\ensuremath{\texttt{count}}\xspace}
\newcommand{\sumfunc}{\ensuremath{\texttt{sum} }\xspace}
\newcommand{\varfunc}{\ensuremath{\texttt{var} }\xspace}
\newcommand{\stdfunc}{\ensuremath{\texttt{std} }\xspace}
\newcommand{\covfunc}{\ensuremath{\texttt{cov} }\xspace}
\newcommand{\corrfunc}{\ensuremath{\texttt{corr} }\xspace}
\newcommand{\medfunc}{\ensuremath{\texttt{median} }\xspace}
\newcommand{\percfunc}{\ensuremath{\texttt{percentile} }\xspace}
\newcommand{\havingfunc}{\ensuremath{\texttt{HAVING} }\xspace}
\newcommand{\selectfunc}{\ensuremath{\texttt{select} }\xspace}
\newcommand{\ratio}{\ensuremath{\rho }\xspace}


\newcommand{\insertion}{\ensuremath{\texttt{INSERT} }\xspace}
\newcommand{\update}{\ensuremath{\texttt{UPDATE} }\xspace}
\newcommand{\delete}{\ensuremath{\texttt{DELETE} }\xspace}

\newcommand{\sysfull}{ActiveClean\xspace}
\newcommand{\sys}{ActiveClean\xspace}
\newcommand{\sysnospace}{ActiveClean}

\newcommand{\tbl}[1]{\textsf{#1}\xspace}
\newcommand{\field}[1]{\textsf{#1}\xspace}
\newcommand{\cost}{\textrm{cost}\xspace}
\newcommand{\ans}{\textsf{ans}\xspace}
\newcommand{\dans}{\Delta\textsf{ans}\xspace}
\newcommand{\cqp}{correction query processing\xspace}
\newcommand{\Cqp}{Correction query processing\xspace}

\newcommand{\reminder}[1]{{{\textcolor{magenta}{\{\{\bf #1\}\}}}\xspace}}
\newcommand{\specialcell}[2][c]{%
  \begin{tabular}[#1]{@{}c@{}}#2\end{tabular}}

\def\ojoin{\setbox0=\hbox{$\bowtie$}%
  \rule[-.02ex]{.25em}{.4pt}\llap{\rule[\ht0]{.25em}{.4pt}}}
\def\leftouterjoin{\mathbin{\ojoin\mkern-5.8mu\bowtie}}
\def\rightouterjoin{\mathbin{\bowtie\mkern-5.8mu\ojoin}}
\def\fullouterjoin{\mathbin{\ojoin\mkern-5.8mu\bowtie\mkern-5.8mu\ojoin}}

%\setlength{\belowcaptionskip}{-10pt}

%\newcommand{\reminder}[1] {}
\pagestyle{plain}

\title{ActiveClean: Progressive Data Cleaning For Advanced Analytics}

%\numberofauthors{1}
%\author{\large Sanjay Krishnan, Jiannan Wang, Michael J. Franklin, Ken Goldberg, Tim Kraska{$\,^\dag$} \\
%\vspace{.2em}\affaddr{\large UC Berkeley, ~~ $^\dag$Brown University} \\
%\vspace{.1em}\affaddr{\large \{sanjaykrishnan, jnwang, franklin, goldberg\}@berkeley.edu}\\
%\affaddr{\large tim\_kraska@brown.edu}
%}

%\fontsize{10pt}{12pt}
%\selectfont

%{\noindent \normalsize \bf Dear SIGMOD Chair and Referees: }

\vspace{.5em}

We thank the reviewers and chair for the very helpful feedback on our paper. 
We addressed all of the concerns and included references to the revised text. 
To summarize the major changes:

\begin{enumerate}
\item Sections 1 and 2 clarify the relationship between \sys and related work in data cleaning that applies machine learning (e.g., \cite{gokhale2014corleone, DBLP:journals/pvldb/YakoutENOI11, yakout2013don}).

\item In Section \ref{dmodel}, we formalize the definition of dirty data and the data cleaning model used in this work.

\item In Section \ref{statements}, we provide problem statements for the two subproblems addressed in this work.

\item Section \ref{arch} presents a revised system architecture that first emphasizes the essential components for correctness, and then highlights optional optimizations. 

\item Section \ref{s:usecase} presents a running example that is referenced in each technical section (Examples \ref{archex}, \ref{upex},\ref{detex1},\ref{detex2},\ref{estex}).

\item We include references to the related work suggested by the reviewing committee \cite{whang2014incremental, papenbrock2015progressive, gruenheid2014incremental, DBLP:journals/pvldb/YakoutENOI11, yakout2013don, heise2014estimating}.

\end{enumerate}
Below we address each reviewer comment in detail:

\vspace{0.5em}

\subsection*{Meta Review Details} 
Machine learning has been studied as a technique to improve the efficiency and reliability of data cleaning~\cite{yakout2013don,gokhale2014corleone}.
In these approaches, machine learning is used to train a Data Cleaning Model (DCM) that learns to predict the value of an incorrect or missing attribute given data that are previously cleaned or known to be clean.
For example, Yakout et al. train a statistical model that evaluates the likelihood of a proposed replacement value \cite{yakout2013don}.
When humans are involved, machine learning can facilitate progressive cleaning by gradually learning a model that predicts the results of the expensive human queries \cite{gokhale2014corleone, DBLP:journals/pvldb/MozafariSFJM14, DBLP:journals/pvldb/YakoutENOI11}.

%This use case leverages clean data to estimate the accuracy of a future repair on dirty data.
%On the other hand, Gokhale et al. \cite{gokhale2014corleone} use machine learning to
%scale crowdsourced data cleaning by learning rules from a small set of examples.
%Crowdsourcing is often expensive and impractical for large datasets. 
%This approach can be coupled with active learning to query a human (or humans) only when the statistical model indicates uncertainty.

\sys addresses the problem of statistical analysis after data cleaning.
We call such models Downstream Statistical Models (DSMs) to contrast with the DCMs in prior work.
An example of a DSM is a movie recommender system that collects dirty user preference data resulting in error-prone predictions.
The DSM is independent of the errors manifest in the data.
The DSM problem is more general than the DCM problem with a broader class of allowed statistical models.
Furthermore, training a DSM on a mix of dirty and clean data can lead to arbitrarily incorrect results (Figure \ref{update-arch1}), and this requires a new algorithm to allow for progressive cleaning while preserving correctness.
%Due to the generality of a DSM, small samples of clean data may not result in a meaningful model.
Finally, we propose several novel extensions to active learning, including dirty data detection and estimation, to clean data that maximally benefit a DSM.
Response \textbf{M5} addresses each of the differences in more detail.


\vspace{0.5em}

\noindent\noindent \textbf{M1. There should be a formal vocabulary introduced early on. The exact idea of ``dirty" here can be hard to follow: what is the exact error type(s) that the system is intended to clean?}

\vspace{0.5em}

We thank the reviewers for this important feedback and clarified that our system applies to data cleaning operations that can be represented as record-by-record mappings.
%This model is sufficiently expressive for our experiments and a number of real-world dirty data scenarios such as making attribute values consistent and missing value filling.
We hope to explore additional data cleaning models that include record deduplication or schema transformations in future work.
We added the following clarification to Section \ref{dmodel}:

\emph{\sys supports data cleaning operations that can be represented as record-by-record transformations.
Formally, there exists a function (implemented via human or algorithm) that when given a dirty record, it will return a unique clean record.
This does not cover errors that simultaneously affect multiple records such as record duplication or schema transformation problems.
We represent this operation as $C(\cdot)$ which can be applied to a record $r$ to recover $r_{clean} = C(r)$.
Therefore, for every $r \in R_{dirty}$ there exists a unique $r' \in R_{clean}$.
We assume that there is a featurization $F(\cdot)$ which is defined over both $R_{dirty}$ and $R_{clean}$ and maps records to tuple of vectors in $(\mathbb{R}^d, \mathbb{R}^l)$ corresponding to features and labels.
So each record corresponds to one training example in the downstream model.}

\vspace{0.5em}

\noindent\textbf{M2. Sections 5-7 are the technical core of the paper, and appear formal at the expense of aiding understanding. They appear to implement something that resembles active learning or bootstrapping, except inside the gradient descent loop. The motivation of some of this is not clear; is it necessary to integrate with the gradient descent? This is not how most active learning methods are implemented. Is it possible to implement these approaches in a way that is orthogonal to the SGD algorithm? The current writeup entangles some of these design choices.} 

We have revised the technical sections of the paper to improve readability.
In Section \ref{correctness}, we present two straight-forward integrations of progressive data cleaning and predictive modeling. 
We explain their limitations and describe how these solutions can result in error-prone models.
To address these limitations, we describe two subproblems: (1) the correctness problem of how to update a dirty model after cleaning and (2) the efficiency problem of how to prioritize cleaning using the downstream model. 
Section \ref{statements} independently formalizes the two problems without reference to Stochastic Gradient Descent.

In Section \ref{model-update}, we propose one solution to problem (1) which updates the model with newly cleaned data using a gradient step.
This can be analyzed as a Stochastic Gradient Descent algorithm, which converges to the true optimum with monotonically decreasing expected error if the gradient steps are unbiased.
We revised the presentation of this section to be independent of the sampling distribution used to select which data to clean.

In Section \ref{dist-samp}, we describes a basic solution to problem (2).
The problem is to find a sampling distribution that maximally reduces the expected error
at each iteration.
While, the solution is derived using optimality w.r.t SGD, this sampling distribution is still beneficial to techniques other than SGD.
The derived optimal distribution is not realizable in practice since it requires knowing the cleaned record value.
Section \ref{dgsample} proposed one solution where the dirty value of the record can be used instead.
Together Section \ref{model-update} and \ref{dist-samp}, provide a minimum viable system that addresses problem (1) and (2).

Section \ref{opti} describes optimizations that improve the convergence rate of the system.
We describe a number of cases when these optimizations are possible.
Our experiments (Section \ref{eval}) evalaute all of these design decisions.

\vspace{0.5em}

\noindent\textbf{M3. In general, the distinction between an ``architecture" and an ``algorithm that fits into the architecture" is quite unclear. The problem with SGD/Active Learning above is one example.}

We have revised the paper to separate problem statements (Section \ref{statements}) and system architecture (Section \ref{arch}), which was part of the confusion in the initial submission.
The architecture in Section \ref{sysover} describes how each of the components fit together and the data flows of the system.
We describe these components in a way that is independent of our algorithmic solutions with Stochastic Gradient Descent and Active Learning.
The new architecture would apply even if the model update problem was addressed with a different algorithm.
We first clarify essential components of the architecture that are needed for a minimum viable solution to problem (1) and (2), 
Then, we highlight the optional components which optimize the execution.
%We also clearly identify the user inputs in Section \ref{uinp}.

\vspace{0.5em}

\noindent\textbf{M4. The paper, and especially the technical sections, would benefit enormously from a detailed running example showing how the algorithm works}

We have added a number of examples at the end of each of the technical sections. Section 4 (Architecture) ends with an intuitive end-to-end running example without technical details (Example \ref{archex}).
Section 5 (Update Problem) ends with an SVM example of how updates are propagated and calculated (Example \ref{upex}).
Section 7.1 (Detection) contains two examples for how the two different types of detectors can be used (Examples \ref{detex1} and \ref{detex2}).
Section 7.2 (Estimation) ends with an example summarizing how linearization could be applied to SVM estimation.

\vspace{0.5em}

\noindent\textbf{M5. Some connections to related work that combines machine learning and data cleaning should be made. See the other reviewers' comments.}

We highlight related work in the background section (Section \ref{alrw}):

\emph{Machine learning can be used as a technique to improve the efficiency and/or reliability of data cleaning\cite{yakout2013don,gokhale2014corleone}.
For example, Yakout et al. train a statistical model that evaluates the likelihood of a proposed replacement value \cite{yakout2013don}.
Another application of machine learning is value imputation, where a missing value is predicted based on those records without missing values.
Machine learning is also increasingly applied to make automated repairs more reliable with human validation \cite{DBLP:journals/pvldb/YakoutENOI11}.
Human input is often expensive and impractical to apply to entire large datasets.
Machine learning can extrapolate rules from a small set of examples cleaned by a human (or humans) to uncleaned data \cite{gokhale2014corleone, DBLP:journals/pvldb/YakoutENOI11}.
This approach can be coupled with active learning \cite{DBLP:journals/pvldb/MozafariSFJM14} to learn an accurate model with the fewest possible number of examples, and intuitively, this means
query a human only when the statistical model indicates uncertainty.\\
The common feature of these approaches is a Data Cleaning Model (DCM) that learns to predict the value of an incorrect or missing attribute given data that are previously cleaned or known to be clean.
In contrast, \sys addresses the problem of statistical analysis, in the form of Machine Learning, on clean data.
We call such models Downstream Statistical Models (DSMs) to contrast them with the DCMs in prior work.
An example of a DSM is a movie recommender system that collects dirty user preference data resulting in error-prone predictions.
The DSM is independent of the errors manifest in the data and is specified by the user.
The DSM problem is more general than the DCM problem with a broader class of allowed statistical models.
There are two key challenges in applying data cleaning before a DSM: (1) statistical validity, and (2) efficiency. 
\sys addresses both of these challenges using an incremental update framework that ensures correctness of intermediate results and several novel extensions to active learning, including dirty data detection and estimation, to clean data that maximally benefit a DSM.}

\vspace{0.5em}
Our related work section~(Section \ref{rw}) highlights the suggested references to progressive data cleaning:

\emph{When data cleaning is expensive, it is desirable to apply it \textbf{progressively}, where analysts can inspect early results with only $k \ll N$ records cleaned.
Progressive data cleaning is a well studied problem especially in the context of entity resolution \cite{altowim2014progressive, whang2014incremental, papenbrock2015progressive, gruenheid2014incremental}.
Prior work has focused on the problem of designing data structures and algorithms to apply data cleaning progressively.
This is challenging because many data cleaning algorithms require information from entire relations.
However, over the last 5 years a number of new results have expanded the scope of progressive data cleaning~\cite{mayfield2010eracer, DBLP:journals/pvldb/YakoutENOI11, yakout2013don}.
\sys explores the statistical implications of using progressive data cleaning before high-dimensional predictive modeling.}

\vspace{0.5em}

\emph{SampleClean\cite{wang1999sample} applies data cleaning to a sample of data, and estimates the results of aggregate queries.
Sampling has also been applied to estimate the number of duplicates in a relation \cite{heise2014estimating}. 
Similarly, Bergman et al. explore the problem of query-oriented data cleaning \cite{DBLP:conf/sigmod/BergmanMNT15}, where given a query, they clean data relevant to that query. 
Existing work does not explore cleaning driven by the downstream machine learning models studied in this work.}

\subsection*{Review 1 Details} 

\noindent\textbf{R1.1: At first, the problem seems a bit too specialized. The abstract is too loaded with technical terms and a turn-off. This is then mitigated in the introduction. \\
As mentioned above, the abstract is (to me) overly technical and did not make me curious. I did not know off the bat what a convex loss model is, what importance sampling is, etc.}

\noindent We revised the abstract to be more accessible:

\emph{Dirty data, including missing, incorrect, or inconsistent values, is an important challenge in data analytics.
Predictive models, such as regression and classification, are increasingly popular forms of data analytics and can be highly sensitive to dirty data.
Although error can be mitigated through data cleaning, it is often very time consuming.
This paper explores techniques to train accurate models without having to clean the entire dataset.
The challenge is that models trained on partially cleaned data can be arbitrarily incorrect requiring a new algorithm for incrementally updating results given newly cleaned data.
We also design sampling algorithms that leverage knowledge about downstream statistical models to prioritize cleaning those records likely to affect the results.
We focus on a popular class of models called convex loss models (e.g., linear regression and SVMs).
The key insight of our framework is that data cleaning can be applied simultaneously with incremental optimization allowing for progressive cleaning while preserving provable properties.
Evaluation on four real-world datasets suggests that for a fixed cleaning budget, \sys returns more accurate models than uniform sampling and Active Learning when corruption is systematic and sparse.}

\vspace{0.5em}

\noindent\textbf{R1.2: Poor embrace of the duplicate detection problem (see details below). Your model of the cleaner seems to preclude any duplicate detection, which certainly cannot happen on individual records. Also you extension for a set of record does not fit the problem of duplicate detection. This is in contrast, for instance, to your ER example in the second column of that page. Appendix A.1 is misleading here, as you mention with Example 7 ``in entity resolution problems..." but do not actually address that problem in the example. Fixing some common inconsistency is not entity resolution.}

We apologize for the confusing terminology and have revised our paper to clarify that we do not address record-level deduplication.
We intended to bring attention to the fact that some types of attribute level inconsistencies are addressed in similar ways to record deduplication.
For example, in our experimental dataset, the inconsistencies ``Pfizer Inc.", ``Pfizer Incorporated", and ``Pfizer" can be addressed using a blocking and matching procedure similar to that used in record deduplication.
That said, we have removed references to entity resolution and described our data cleaning model in more precise terms.

\vspace{0.5em}

\noindent\textbf{R1.3: Cheated by using a narrower font than required. Will have trouble with camera ready copy if publisher insists on proper font.\\
- I would not use ``overview" as a verb...
- 3.2: the detector select -> the detector selects
- 4.3: Wrong quotation marks around ``learning".
- QED symbols on page 8 are ugly when placed directly after formula. 
- References need a clean up. Just as an example: Venue is missing for [24], year is mentioned 3 times for [8], [11], etc. Page numbers appear sporadically.}

\noindent We have addressed all of the formatting and copy editing issues.

\vspace{0.5em}

\noindent\textbf{R1.4:There is some related work specifically addressing progressive/incremental entity resolution. You might want to point your readers to this.
\\E.g.
\\- Incremental entity resolution on rules and data, Whang et al. VLDB Journal 2014
\\- Progressive duplicate detection, Papenbrock et al., TKDE 2015
\\- Incremental record linkage, Gruenheid et al., PVLDB 2014
\\- Another work that is related is ``Estimating the Number and Sizes of Fuzzy-Duplicate Clusters" by Heise et al. CIKM 2014, which also incrementally cleans samples of data to predict in this case the number of record matches.}

Thank you for highlighting these references, and we have included them in our related work~(see response \textbf{M5}).

\vspace{0.5em}

\noindent\textbf{- Page 1, last paragraph in column 1 reads as if reference to [3] is a reaction to the work referenced in the previous sentence, i.e., the term ``remains" is misleading.
- I did not quite understand the short paragraph on crowd-sourcing. Why is this even relevant?
 I believe it would suffice to simply state that cleansing is expensive...}

We appreciate the thorough feedback and have tightened up the writing in the introduction. In particular, we have consolidated the motivation to a single paragraph describing the expense of data cleaning. We include a single sentence reference to related work on crowdsourcing/human-guided data cleaning which we believe is relevant due to the use of active learning.


\subsection*{Review 2 Details}

\noindent\textbf{R2.1: The definition of ``clean data" is imprecise and not clear. It appears that ``cleaning" in this system refers to entity resolution, cleaning w.r.t. dependencies, and possibly other actions as needed by the application. This makes it difficult to gauge overall accuracy when there are different interpretations of cleanliness. It is not clear how entity resolution and cleaning w.r.t. dependencies can be done holistically.}

We addressed this concern in response \textbf{M1}.

\vspace{0.5em}

\noindent\textbf{R2.2: The paper describes a problem setting focused on modelling the iterative cleaning process rather than actual data management problems. The paper may be better suited at an ML venue.}

Over the last 5 years a number of new results have expanded the scope of progressive and interactive data cleaning~\cite{mayfield2010eracer, DBLP:journals/pvldb/YakoutENOI11, yakout2013don, altowim2014progressive, whang2014incremental, papenbrock2015progressive, gruenheid2014incremental}.
However,  it turns out that the straight-forward application of existing progressive data cleaning methods can lead to error-prone and misleading results (Section \ref{correctness}).
Recognizing that data analytics is increasingly moving towards statistical modeling, \sys presents an initial exploration of this problem.  

\vspace{0.5em}

\noindent\textbf{R2.3: Missing references to related work on interactive data cleaning. For the comparative evaluation, 2/3 techniques are ML based techniques, not interactive data cleaning systems. See D2.\\
D2: Data cleaning systems have also considered interactive engagement with the user and the application of ML techniques. 
i) Mohamed Yakout, Laure Berti-Equille, Ahmed K. Elmagarmid. Don't be SCAREd: use SCalable Automatic REpairing with maximal likelihood and bounded changes. SIGMOD Conference 2013: 553-564
ii) Mohamed Yakout, Ahmed K. Elmagarmid, Jennifer Neville, Mourad Ouzzani, Ihab F. Ilyas.
Guided data repair. PVLDB 4(5): 279-289 (2011).
}

In Section \ref{alrw}, we contrast two applications of machine learning in data cleaning: Data Cleaning Models (DCMs) and Downstream Statistical Models (DSMs).
A Data Cleaning Model (DCM) is a model that learns to predict the value of an incorrect or missing attribute given data that are previously cleaned or known to be clean.
DCMs are used to improve the efficiency or reliability of data cleaning by: extrapolating rules from a small number of cleaned examples, estimating likelihoods that a repair is accurate, or numerical value imputation.
In contrast, the DSM problem is an analyst-specified model to be trained after data cleaning.
The DSM problem is more general, and as a result, a number of the optimizations used in the DCM literature do not apply.
Furthermore, training a DSM on a mix of dirty and clean data can lead to arbitrarily incorrect results (Figure \ref{update-arch1}), and this requires a new algorithm to allow for progressive cleaning while preserving correctness.
\sys is an estimation framework for DSMs using existing data cleaning methods and not a new data cleaning algorithm.
In our experiments (Section \ref{comp}), we illustrate the relative contribution of different components in \sys, which benchmarks the entire framework against a minimal solution that still provides correct results.


\vspace{0.5em}

\textbf{R2.4: Sampling is an important part of the framework and influences the accuracy of the cleaning. Yet, there is little discussion on sampling rate, or how a sample is chosen.}

We revised Sections 6 and 7 to be more precise about the sampling.
Section 6 describes sampling without estimation or detection:

\emph{The model update received a sample with probabilities $p(\cdot)$.
\sys uses a sampling algorithm that selects the most valuable records to clean with higher probability. }

\vspace{0.5em}

Section 7 describes how sampling can be improved with estimation and detection and intuition on why those optimizations improve result accuracy.

\vspace{0.5em}

\textbf{R2.5: An end-to-end running example in Section 5 is needed to highlight the intuition of the cleaning process.}

We addressed this point with a number of examples (see response \textbf{M4}).


\vspace{0.5em}


\subsection*{Review 3 Details}
\noindent\textbf{R3.1: The authors do not distinguish between the system architecture and the individual issues that they are presenting.}

Response \textbf{M3} describes several revisions to the architecture including: separating problem formalization and architecture, discussing the data flow rather than the algorithms, and highlight essential components for correctness versus optimizations.

\vspace{0.5em}

\noindent\textbf{R3.2: The paper uses lots of definitions, and a multitude of that do not necessarily contribute to readability.
Without being an expert in the field, I found it extremely difficult to follow the paper as it touches upon multiple problems at the same time: data cleaning, model training, convex analytics, etc., uses definitions, notation and lots of examples that did not allow me to have a global understanding of the work.\\
I would prefer to have a more focused paper on one of these aspects that has concrete goals and then, having an overview of the architecture of the system as a small section. I believe that the architecture should not be the focus and the skeleton of this paper. Instead, I believe that the authors could focus on the individual problems.}

We have discussed a number of specific textual revisions in response \textbf{M2} and \textbf{M3}. Here are a list of other revisions to improve the readability:

\begin{enumerate}
\item We have expanded the background section to provide a more detailed setup and context to the problem.
\item Our problem formulation is now divided into two subproblems: (1) correctness, and (2) efficiency.
\item We revised the technical sections to first present a minimum viable solution that addresses the two subproblems (Section \ref{model-update} and Section  \ref{dist-samp}).
\item The next section (Section \ref{opti}) describes optional optimizations that can be applied in a number of practical cases.
\item Detailed derivations are now in the appendix, and the additional space has been used for three new examples in the technical sections (Sections \ref{model-update}-\ref{opti}).
\end{enumerate}


\maketitle

\begin{abstract}
Data management systems are increasingly supporting advanced analytics such as Machine Learning.
A perennial challenge is the presence of dirty data -- in the form of missing, duplicate, incorrect or inconsistent values -- which can bias the learning algorithms, leading to incorrect or error-prone predictions. Although, this can be ameliorated by cleaning the input dataset using existing data cleaning techniques, the cost of doing so (e.g., manually finding and cleaning all data) is practically and economically infeasible.  In addition, existing learning algorithms are not suited for training on partially cleaned data, which can lead to dramatically incorrect models. In response, we have developed the ActiveClean system, which integrates incremental data cleaning with machine learning algorithms.  ActiveClean iteratively updates a machine learning model as data records are individually (or in batches) cleaned.  
We additionally study numerous database and ML-oriented optimizations to speed the end-to-end process. Evaluation on four real-world datasets suggests that our methodology can learn more accurate models by cleaning significantly fewer records (up-to an order of magnitude) than alternatives such as uniform sampling and active learning, thus bridging the gap between the reality of dirty data sources and the quality of predictive models at the end of the pipeline.
\end{abstract}

\if{0}
\begin{abstract}
Databases are susceptible to various forms of corruption, or \emph{dirtiness}, such as missing, incorrect, or inconsistent values.
Increasingly, modern data analysis pipelines involve Machine Learning for predictive models which can be sensitive to dirty data.
Dirty data is often expensive to repair, and naive sampling solutions are not suited for training high dimensional models.
In this paper, we propose \sysfull, an anytime framework for training Machine Learning models with budgeted data cleaning.
Our framework updates a model iteratively as small samples of data are cleaned, and includes numerous optimizations such as importance weighting and dirty data detection.
We evaluate \sys on 4 real datasets and find that our methodology can return more accurate models for a smaller cost  than alternatives such as uniform sampling and active learning.
\end{abstract}
\fi

\setcounter{page}{1}

\section{Introduction}
Large and growing data are often succeptible to various forms of corruption, or \emph{dirtiness}, such as missing, incorrect, or inconsistent values.
These corruptions can negatively affect subsequent analysis in subtle but significant ways.
While cleaning these corruptions is often highly beneficial and there are numerous established techniques, data cleaning still poses a significant analysis bottleneck \cite{khayyat2015bigdansing,sampleclean, chu2015katara}.
Cleaning large data can be expensive, both computationally and in human effort, as an analyst has to program repairs for all errors manifest in the data \cite{kandel2012}.
In some applications, simple data transformations may not be reliable necessitating the use of even costlier machine learning or crowdsourcing \cite{gokhale2014corleone,park2014crowdfill}.

To cop

\section{Background}
\subsection{Machine Learning and Loss Minimization}
In parametric Machine Learning, the goal is to learn a set of model \emph{parameters} $\theta$ from training examples.
A common theoretical framework in Machine Learning is empirical risk minimization.
We start with a set of training examples $\{(x_{i},y_{i})\}_{i=1}^{N}$
on which we minimze an loss function $\phi$ at each point parametrized that is parametrized by $\theta$.
\[
 \theta^{*}=\arg\min_{\theta}\sum_{i=1}^{N}\phi(x_{i},y_{i},\theta)
\]
For example, in a linear regression $\phi$ is:
\[
\phi(x_{i,}y_{i},\theta) = \|\theta^Tx_{i} - y_i \|_2^2
\]
$\phi$ is often designed to be \emph{convex}, essential meaning bowl-shaped, to make the training this model 
tractable.
This class of problems includes all generalized linear models and support vector machines.

Typically, a \emph{regularization} term $r(\theta)$ is added to this problem.
The regularization term $r(\theta)$ is traditionally what is used to increase the robustness of the model.
$r(\theta)$ penalizes high or low values of feature weights in $\theta$ to avoid overfitting to noise in the
training examples.
\[
 \theta^{*}=\arg\min_{\theta}\sum_{i=1}^{N}\phi(x_{i},y_{i},\theta) + r(\theta)
\]
For example, a popular variant of linear regression is called LASSO which is:
\[
 \theta^{*}=\arg\min_{\theta}\sum_{i=1}^{N}\|\theta^Tx_{i} - y_i \|_2^2 + \lambda \cdot \|\theta\|_1
\]
By applying the L1 regularization term, if one of the features is particularly noisy, and does not add predictive value, it will get excluded.

Along these lines, other robust techniques have also been proposed.
For example, in the case of linear regression, we can change the $L_2$ norm to an $L_1$ norm to mitigate the effect of outliers:
\[
\phi(x_{i,}y_{i},\theta) = \|\theta^Tx_{i} - y_i \|_1
\]
The quadratic L2 loss implies that examples that deviate far from the typical example are quadratically penalized as opposed to linearly penalized with the L1 loss.
There is a natural tradeoff between robustness and efficiency.
The more robust a technique is, the less efficient it will be (i.e. estimate variance for a fixed number of training examples).

\subsection{Training on Clean Data}
How exactly does a data cleaning approach differ from a robust method?
We often in the problem setting where our data is systematically biased.
Consider an image classification task with incorrect labeling.
If we train a model with respect to the incorrect labels, while we might have achieve a good out-of-sample accuracy on the incorrect labels, the classifications are incorrect in a semantic sense.
Likewise, consider the case where we are predicting future product demand based on corrupted historical data.
Training a model with respect to the corrupted data might have a low $R^2$ cross-validation error, but is incorrect
at predicting the future trends.
In such scenarios, we see data cleaning as complementary to robust statistics.
Data cleaning gives us information about the ``true" data distribution, which is highly beneficial when errors have systematic biases.
Without cleaning, certain subpopulations of data might be frequently mispredicted. 

\subsection{SampleClean Project}
In our previous work, we studied the relationship between approximate query processing, data cleaning, and sampling \cite{wang1999sample, technicalReport}.
Traditionally, data cleaning has explored expensive, up-front cleaning of entire datasets for increased query accuracy.
We proposed the SampleClean problem, in which an analyst cleans a small sample of data, and then estimates the result to an aggregate query e.g., \sumfunc, \countfunc, or \avgfunc.
The main insight from the SampleClean project is that highly accurate answers for aggregate queries does not require cleaning the full dataset.
Generalizing this insight, there is a deep relationship between the application (i.e., the query) and how an analyst should budget their effort in data cleaning.
In fact, \avgfunc and \sumfunc queries are a special case of the convex loss minimization discussed in the previous section:
\[
\phi(x_{i,}y_{i},\theta) = (x_{i} - \theta)^2
\]

We then extended the SampleClean work to study cleaning Materialized Views \cite{technicalReport}.
Suppose base data is updated with insertions, deletions, or updates, we explored how we could efficiently propagate
changes to a sample of the view instead of the full view.
Subsequent queries on the view could be answered approximate.

We see this line of work as an extension and generalization of the work that we did in the past.
There are several new contributions in this work.
First, Machine Learning models give us a richer geometric information (e.g., gradients), which we can use 
prioritizing sampling.
In this work, we explore non-uniform sampling as opposed to the uniform sampling methodologies studied before.
Next, machine learning models are trained iteratively making them more amenable to adaptive processing where cleaned data can be fed back to inform the next set of samples.
Finally, in this work, we give a deeper treatment to the problem of sparsity where only small clustered subsets of data are corrupted.
We address this by using intelligent partitioning that segregates clean and dirty data.

\subsection{Stochastic Gradient Descent}
Sampling is naturally a part of any Machine Learning workflow 

Our goal is to design an anytime framework for training models on dirty data.
Our main idea is to introduce data cleaning into the iterative algorithms that train the models. 
The problems described in the previous subsections are often trained using a technique called Stochastic Gradient Descent (SGD) or one of its variants.
The basic idea of SGD is to draw a data point at random, calculate the gradient at that point, and then update a current best estimate with that gradient.
\[
\theta^{(t+1)}\leftarrow\theta^{(t)}-\gamma\nabla\phi(x_{i},y_{i},\theta^{(t)})
\]
 SGD can also be applied in a ``mini-batch" mode, where we draw a subset of data at random and update with the average gradient.
 \[
 \theta^{(t+1)}\leftarrow\theta^{(t)}-\gamma\sum_{i\in S^{(t)}}\nabla\phi(x_{i},y_{i},\theta^{(t)})
 \]
 At a high-level, this problem mirrors the problem explored in this work. We start with an initialization (the dirty model) $\theta^{(0)}$ and iteratively update the existing best model as we get more clean data. 
SGD and its variants are well-studied and there are lower-bounds on the convergence rates using these techniques. 

Recently, a number of works have explored non-uniform sampling distributions for stochastic optimization \cite{zhao2014stochastic, qu2014randomized}.
The main insight is that non-uniform distributions may on average estimate the gradient accurately.
The technique that is applied is called importance sampling.
Importance sampling preserves the exepected value of a parameter while trying to sample from a distribution that results in lower variance.





 
\section{System Architecture}\label{arch}
Now, we overview the \sys architecture, its goals, and describe how our framework integrates with existing data cleaning solutions.

\subsection{Overview}
The primary goal of \sys is to provide a framework that wraps around existing Machine Learning and Data Cleaning systems.
Instead of cleaning the entire data upfront and then training a model, we initialize the framework with a model, and iteratively correct it with small batches of clean data.
Built into the framework are feedback mechanisms that use information from the cleaning to improve performance on future batches of clean data.

There are two basic operating modes of \sys: (1) error detection oracle and (2) adaptive error detection. 
These two modes differ in the way they partition dirty and clean data, where (1) assumes we can differentiate the two in advance, and (2) assumes we can learn it from samples.
Partitioning serves two purposes: (1) it reduces the variance of our updates because we can cheaply scan over data we know that is clean, and (2) it increases the fraction of actually dirty records in the candidate batch.
A good example of why we need the second objective is seen in the context of crowdsourcing.
If we have a crowdworker clean records, we will have to pay them for the task whether or not the record required cleaning.
Partitioning has important ramifications since classifiying erroneous data as cleaned can impart a bias on our model.
We will overview how the architecture and applications change based on these modes.

First, we overview our notation and terminology:

\vspace{0.5em}

\noindent \textbf{Featurization: } Given a relation $R$ with a set of attributes $A$.
There is a featurization function $F$ which maps every row in $\mathcal{R}$ to a $d$ dimensional feature vector and a $l$ dimensional label tuple: 
\[F(r \in \mathcal{R}) \mapsto (\mathbb{R}^l, \mathbb{R}^d)\]
The result of the featurization are the data matrices $X$ and $Y$.
\[
F(R)\rightarrow (X,Y)
\]
We consider problems in which the training examples (i.e., rows in the data matrix) have a one-to-one relationship with rows in the base data ($R$).

\vspace{0.5em}

\noindent\textbf{Initialization: } To initialize \sys, the user gives us the base dirty relation $R$, the featurization $F$, and a dirty model $\theta^{(d)}$ that is derived from the dirty relation. So initially $\theta^{(1)} = \theta^{(d)}$

\vspace{0.5em}

\noindent\textbf{Error Detection: } The first step in \sys is error detection. In this step, we select a candidate set of dirty records $R_{dirty} \subseteq R$.  

\vspace{0.5em}

\noindent\textbf{Error Sampling: } The second step in \sys is sampling. Since we cannot clean all of the dirty records, we take a sample of the records $S_{dirty} \subseteq R_{dirty}$.

\vspace{0.5em}

\noindent\textbf{Error Repair: } Next, we apply a repair procedure $C$ to the sample of data $S_{dirty}$ to get a clean sample $S_{clean}$. Given a dirty record $r$, the error repair module applies a repair $C$ to the record, to return $C(r)\mapsto r_{clean}$. 

\vspace{0.5em}

\noindent\textbf{Model Update: } Then, we update the model $\theta^({(i)})$ based on the newly cleaned data $F(S_{clean})$. The result is the updated model $\theta^({(i+1)})$.

\vspace{0.5em}

\noindent\textbf{Error Impact Estimate: } Based on the change between $F(S_{clean})$ and $F(S_{dirty})$, we direct the next iteration of sampling to select points that will have changes most valuable to the next model update.

\subsubsection{Error Detection Oracle}
For many types of errors, it is possible to efficiently enumerate a set of corrupted records and enumerate what is wrong with these records.
For example, in example \ref{exm-1}, this is the set of records with missing values and we know exactly which attributes (and corresponding features) are corrupted.
We describe these types of errors as having an \emph{error detection oracle}.
In Figure \ref{sys-arch4a}, we add more detail to our introductory architecture diagram when we are applying \sys to such errors.

\begin{figure}[t]
\centering
 \includegraphics[width=0.6\columnwidth]{figs/arch1a.png}
 \caption{We detail the \sys architecture when we are using an error detection oracle. In this model, we have a way to enumerate all the records that are dirty before running any cleaning. \label{sys-arch4a}}
\end{figure}

\begin{enumerate}
\item Initialize with a dirty model $\theta^{(1)} = \theta^{(d)}$, batch size $b$, and number of iterations $T$
\item Error Detection. In the first step, we use the oracle to select a set of records $R_{dirty}$.
Associated with each $r \in R_{dirty}$ is a set of errors $e_r$ which tells us which features are corrupted.
For example, if a missing value in attribute $a$ affects features $\{1,2,6\}$, then $e_r=\{1,2,6\}$.
The error detection step gives us the following tuple: $(R_{dirty},E_r)$.

\item For each iteration $i=1,...,T$

\begin{enumerate}
\item Error Sampling. We sample a batch of records from $R_{dirty}$ each record has a sampling probability $p_i$. (Described in Section \ref{model-update})
\item Error Repair. We apply the user-specified data cleaning $C(S_{dirty})$
\item Model Update. We use the results of the cleaning to update the model $\theta^{(i+1)}$ (Described in Section \ref{model-update}).
\item Error Impact Estimate. Based on the results of the cleaning, we update $p_i$ to guide the data cleaning to the most valuable data (Described in Section \ref{sampling}).
\end{enumerate}
\item Return $\theta^{(T+1)}$
\end{enumerate}

\noindent We highlight example use cases of this architecture using data cleaning methodologies proposed in the liteature.

\vspace{0.5em}

\noindent\textbf{Constraint-based Repair: }
One model for handling errors in database declaring a set of constraints on a database and 
iteratively fixing records such that the constraints are satisfied \cite{DBLP:journals/pvldb/YakoutENOI11, DBLP:journals/pvldb/FanLMTY10, khayyat2015bigdansing}.
However, automatically repairing can be unreliable \cite{DBLP:journals/pvldb/FanLMTY10}.
Recently proposed systems like Guided Data Repair \cite{DBLP:journals/pvldb/YakoutENOI11}, use human input to validate suggested fixes.

\vspace{0.5em}

\emph{Error Detection. } Let $\Sigma$ be a set of constraints on the relation $\mathcal{R}$. 
In the error detection step, we select a subset of records $\mathcal{R}_{dirty} \subseteq \mathcal{R}$ that violate at least one constraint.
The set $e_r$ is the set of columns for each record which have a constraint violation.

\vspace{0.5em}

\emph{Error Repair. } \sys supports both automated and human validated data repair for constraints. For automated techniques, we can apply the record-level ``local" repair proposed in \cite{DBLP:journals/pvldb/FanLMTY10} or we can use human corrections as in \cite{DBLP:journals/pvldb/YakoutENOI11}. 

\begin{example}
Consider our EEG data running example, for an example of constraint-based cleaning.
We can add a constraint that one of the electrical signals cannot be greater than 4V.
For all records whose value is above 4V, we would select them in the error detection step.
Then, in the error repair step, we could apply a repair that sets the erroneous signal value to its most likely value.
\end{example}

\vspace{0.5em}

\noindent\textbf{Entity Resolution: }
Another common data cleaning task is Entity Resolution \cite{gokhale2014corleone, DBLP:journals/pvldb/KopckeTR10, wang2012crowder}.
A common pattern in Entity Resolution is to split up the operation into two steps: blocking and matching.
In blocking, attributes that should be the same are coarsely grouped together.
In matching, those coarse groups are resolved to a set of distinct entities.
Increasingly the matching step is done by crowd workers \cite{wang2012crowder, gokhale2014corleone}, leading to very expensive costs.

\vspace{0.5em}

\emph{Error Detection. } This is the matching step. Let $S$ be a similarity function that takes two records and returns a value in $[0,1]$ (1 most similar and 0 least similar). For some threshold $t$, $S$ defines a similarity relationship between two records $r$ and $r'$:
\[
r \approx r' : S(r,r') \ge t
\] 
In the error detection step, $R_{dirty}$ is the set of records that have at least one other record in the relation that satisfy $r \approx r'$.
The set $e_r$ is the attributes of $r$ that have entity resolution problems.

\vspace{0.5em}

\emph{Error Repair. } \sys supports both automated and human validated entity resolution for resolving a set of matches to a single entity.

\begin{example}
An example of an Entity Resolution problem is where the patient's gender is inconsistently represented (e.g., ``Male", ``M", ``Man"). 
We can define a similarity relationship $(\text{first char =} 'M')$ and select all records that satisfy this condition.
In the error repair step, a human can fix the inconsistencies setting all records that meet the condition to the canonical value.
\end{example}

\subsubsection{Adaptive Detection}
The second operating mode of \sys is adaptive detection.
In this mode, the analyst does not know the set of errors in advance but learns them as she cleans more data.
Consider the case when she has to do exploratory data analysis to discover the errors.
In Figure \ref{sys-arch4b}, we add more detail to our introductory architecture diagram when we are applying \sys to such errors.

\begin{figure}[t]
\centering
 \includegraphics[width=0.6\columnwidth]{figs/arch1b.png}
 \caption{We detail the \sys architecture when we are using adaptive detection. In this model, we do not assume that we can enumerate the set of erroneous records in advance. \label{sys-arch4a}}
\end{figure}

\begin{enumerate}
\item Initialize with a dirty model $\theta^{(1)} = \theta^{(d)}$, batch size $b$, number of iterations $T$, and $R_{dirty} = R$
\item For each iteration $i=1,...,T$
\begin{enumerate}
\item Error Sampling. We sample a batch of records from $R_{dirty}$ each record has a sampling probability $p_i$. (Described in Section \ref{model-update})
\item Error Repair. We apply the user-specified data cleaning $C(S_{dirty})$
\item Model Update. We use the results of the cleaning to update the model $\theta^{(i+1)}$ (Described in Section \ref{model-update}).
\item Error Impact Estimate. Based on the results of the cleaning, we update $p_i$ to guide the data cleaning to the most valuable data (Described in Section \ref{sampling}).
\item We maintain a classifier to prune the set of dirty records based on what was cleaned.
\end{enumerate}
\item Return $\theta^{(T+1)}$
\end{enumerate}

\noindent We highlight an example use case of this architecture.

\vspace{0.5em}

\noindent\textbf{Interactive Data Cleaning with OpenRefine.}
\begin{example}
Consider our analyst using an interative tool such as OpenRefine \cite{openrefine} to clean her data. 
She takes a sample of data from the entire dataset.
She then uses the tool to determine which records are corrupted and dirty.
As she marks more records as dirty, the system learns what features determines a dirty record.
This classifier can be used to guide cleaning in future batches of data.
\end{example}

\iffalse
\subsection{Optimizations}
There are three aspects of \sys, that allow us to achieve this design point: error partitioning, gradient-based model update (Section \ref{model-update}), estimate-driven sampling (Section \ref{sampling}).

\vspace{0.5em}

\noindent\textbf{Partitioning Dirty and Clean Data: } In many applications, enumerating the set of corrupted records is much easier than cleaning them. For example, we may be able to select the set of rows that have missing values but actually filling those missing values is expensive. Likewise, in the constraint literature, selecting a set of rows that have a violated constraint can be done in polynomial time, however, fixing the constraints is NP-Hard.
In our error detection step, we partition the dirty and clean data.
Partitioning serves two purposes: (1) it reduces the variance of our updates because we can cheaply scan over data we know that is clean, and (2) it increases the fraction of actually dirty records in the candidate batch.
A good example of why we need the second objective is seen in the context of crowdsourcing.
If we have a crowdworker clean records, we will have to pay them for the task whether or not the record required cleaning.
To efficiently use this partitioning, we need a database solution indexing dirty and clean data.

\vspace{0.5em}

\noindent\textbf{Gradient-Based Updates: } In \sys, we start with a dirty model and then make an update using a gradient step. Here, we can draw an analogy to Materialized View maintenance, since after all, a model parametrized by $\theta$ is just a table of floating point numbers.
Krishnan et al. proposed a technique called sample view cleaning, in which they take a clean sample of data and propagate the updates to a Materialized View.
Similarly, in this work, we take the information from a sample of cleaned data and propagate an update with the gradient.

\vspace{0.5em}

\noindent\textbf{Estimate-Driven Sampling: } Repair is the most expensive step in the workflow, so optimizing for scan cost may lead to negligible overall time improvements.
We can sacrifice a small overhead in pre-computation for each data point to determine its value to the model and select a sampling distribution accordingly.
Intuitively, while each iteration has an increased cost, it also makes more progress towards the optimum.
\fi



\section{Detection}\label{det}
To maximize the benefit of data cleaning, when we sample data to clean, we want to ensure that the data that we are sampling is likely to be dirty.

\subsection{Goals}
The detection component needs to give us two important pieces of information about a record: (1) whether the record is dirty, and (2) if it is dirty, what is wrong with the record.
From (1), we can select a subset of dirty records to sample at each batch. 
(2) is important so we can estimate how valuable cleaning will be for this record.
There are two cases that we explore in \sys: \emph{a priori} and \emph{adaptive}.
In the a priori case, we recognize that for many data cleaning methodologies, we can efficiently select the set of dirty records without repair.
In the adaptive case, we relax this assumption, and explore how we can learn which records are dirty and clean with a classifier.

\subsection{A Priori Case}
For many types of dirtiness such as missing attribute values and constraint violations, it is possible to efficiently enumerate a set of corrupted records and enumerate what is wrong with these records.

\begin{definition}[A Priori Detection]
Let $r$ be a record in $R$. An a priori detector is a detector that returns a Boolean of whether the record is dirty and a set of columns $e_r$ that are dirty.
\[
D(r) = (\{0,1\}, e_r)
\]
From the set of columns that are dirty, we can find the corresponding features that are dirty $f_r$ and labels that are dirty $l_r$.
\end{definition}

\noindent We highlight example use cases of this definition using data cleaning methodologies proposed in the liteature.

\vspace{0.5em}

\noindent\textbf{Constraint-based Repair: }
One model for handling errors in database declaring a set of constraints on a database.
Data are cleaned iteratively until the constraints are satisfied \cite{DBLP:journals/pvldb/YakoutENOI11, DBLP:journals/pvldb/FanLMTY10, khayyat2015bigdansing}.

\vspace{0.5em}

\emph{Detection. } Let $\Sigma$ be a set of constraints on the relation $\mathcal{R}$. 
In the detection step, we select a subset of records $\mathcal{R}_{dirty} \subseteq \mathcal{R}$ that violate at least one constraint.
The set $e_r$ is the set of columns for each record which have a constraint violation. 

\begin{example}
An example of a constraint for our running example dataset is that the \texttt{status} of
a contribution can be only ``covered" or ``non-covered".
Any other value for \texttt{status} is an error.
\end{example}

\vspace{0.5em}

\noindent\textbf{Entity Resolution: }
Another common data cleaning task is Entity Resolution \cite{gokhale2014corleone, DBLP:journals/pvldb/KopckeTR10, wang2012crowder}.
Entity Resolution is the problem of standardizing attributes that represent the same real world entity.
A common pattern in Entity Resolution is to split up the operation into two steps: blocking and matching.
In blocking, attributes that should be the same are coarsely grouped together.
In matching, those coarse groups are resolved to a set of distinct entities.

\vspace{0.5em}

\emph{Detection. } This is the matching step. Let $S$ be a similarity function that takes two records and returns a value in $[0,1]$ (1 most similar and 0 least similar). For some threshold $t$, $S$ defines a similarity relationship between two attributes $r(a)$ and $r'(a)$:
\[
r(a) \approx r'(a) : S(r(a),r'(a)) \ge t
\] 
In the detection step, $R_{dirty}$ is the set of records that have at least one other record in the relation that satisfy $r(a) \approx r(a)'$.
The set $e_r$ is the attributes of $r$ that have entity resolution problems.

\begin{example}
An example of an Entity Resolution problem is seen in our earlier example about corporation names e.g. ``Pfizer Inc.", ``Pfizer Incorporated", ``Pfizer".. 
We can define a similarity relationship $WeightedJaccard(r1,r2)>0.8$ and select all records that satisfy this condition (their Weighted Jaccard Similarity is greater than 0.8).
\end{example}

\subsection{Adaptive Detection}
Next, we explore how we can handle the case where the detection is not known in advance.
Based on what we have already cleaned, we can learn a detection classifier (note that this ``learning" is distinct from the ``learning" at the end of the pipeline).
The challenge in formulating this problem is that we not only need to know whether or not a record is dirty, but also how it is dirty (e.g. $e_r$ in the a priori case).
Instead of assuming that we know which features are corrupted, let us say that we know that there are $u$ classes of data corruption.
These classes are corruption categories that do not necessarily align with features, but every records is classified with at most one category.
For example, suppose we have outliers and missing values, there are three classes of corruption: outliers, missing values, and both.
As the analyst cleans data, she tags dirty data with one of the $u$ classes.
Then, the detection problem reduces to a multiclass classification problem.
To address this problem, we can use any multiclass classifier, and we use an all-versus-one SVM in our experiments.
Since this classifier is internal to our system, it does not have to be a convex model (i.e., it can be a Decision Tree or Random Forest).

When an example $(x,y)$ is cleaned, the repair step also has to provide a label to which of the ${\text{clean}, 1,2,...,u}$ classes it belongs. It is possible that $u$ increases each iteration as more types of dirtiness are discovered. 
Thus, after cleaning $k$ records, we have a dataset of $k$ records labeled with $u+1$ classes (including one for ``not dirty").

\begin{definition}[Adaptive Case]
To select $R_{dirty}$, we select the set of records for which $\kappa$ gives a positive error classification (i.e., one of the $u$ error classes).
After each sample of data is cleaned, the classifier $\kappa$ is retrained.
So the result is:
\[D(r) = (\{1,0\},\{1,...,u+1\})\]
\end{definition}

We highlight an example of adaptive detection using an interative data cleaning tool such as OpenRefine \cite{openrefine}.

\vspace{0.25em}

\noindent\textbf{Interactive Data Cleaning.}
\begin{example}
OpenRefine is a spreadsheet-based tool that allows users to explore and transform data.
However, it is limited to clean data that can fit in memory on a single personal computer.
Since the cleaning operations are coupled with data exploration, we do not know what is dirty in advance (the analyst may discover new errors as she cleans).

Suppose our analyst wants to use OpenRefine to clean our running example dataset with \sys.
She takes a sample of data from the entire dataset and uses the tool to discover errors.
For example, she finds that some drugs are incorrectly classified as both drugs and devices.
She then clears the device attribute for all records that have the drug name in question.
Every time she makes a batch data transformation (i.e., clearning the device attribute), we can list the set of records that have changed.
Each transformation becomes and error class, and the records that have changed records become positive training examples for a classifier to guide future samples.
\end{example}
\section{Optimizations}\label{opti}


In this section, we describe two approaches to optimization, the {\it Detector} and the {\it Estimator}, that
improves the efficiency of the cleaning process.  
Both approaches are designed to increase the likelihood that the 
{\it Sampler} will pick dirty records that, once cleaned,
most move the model towards the true clean model.


\section{Dirty Data Detection}\label{det}
If corrupted records are relatively rare, the batch sampling might be very inefficient.
The analyst may have to sample many batches of data before finding a corrupted record.
In this section, we describe how we can couple \sys with prior knowledge about which data are likely to be dirty.
In the data cleaning literature, error detection and error repair are two distinct problems~\cite{DBLP:series/synthesis/2012Fan, Dasu:2003:EDM:861869, rahm2000data}.
Error detection is often considered a substantially easier, since one can often declare a set of integrity rules on a database (e.g., an attribute must not be NULL), and select rows that violate those rules.
On the other hand, repair is harder and often requires human involvement (e.g., imputing a value for the NULL attribute).

\subsection{Detection Problem}
First, we describe the required properties of the dirty data detector.
The detector returns two important aspects of a record: 
(1) whether the record is dirty, and (2) if it is dirty, on which attributes there are errors.
The sampler can use (1) to select a subset of dirty records to sample at each batch and 
the estimator can use (2) to estimate the value of data cleaning based on other records with the same corruption.

\begin{definition}[Detector]
Let $r$ be a record in $R$. An detector is a function that returns a Boolean of whether the record is dirty and a set of columns $e_r$ that are dirty.
\[
D(r) = (\{0,1\}, e_r)
\]
From the set of columns that are dirty, find the corresponding features that are dirty $f_r$ and labels that are dirty $l_r$.
\end{definition}

We will consider two types of detectors: exact rule-based detectors that rely on detecting integrity constraint or functional dependency violations, and approximate adaptive detectors that learn which data are likely to be dirty.

\subsection{Rule-Based Detector}\label{rule-det}
Data quality rules are widely studied as a technique for detecting data errors.
In most rule-based frameworks, an analyst declares a set of rules $\Sigma$ and checks whether a relation $R$ satisfies such rules.
These rules can be declared in advance before applying \sys, or constructed from the first batch of sampled data.
This paper focuses on integrity constraints (ICs), conditional functional dependencies (CFDs), and matching dependencies (MDs) where it is efficient to enumerate the set of records that violate at least one declared rule. 

Let $R_{viol}$ and $R_{sat}$ be the subset of records in $R_{ditry}$ that violate at least one rule and satisfy all rules respectively.
The rule-based detector modifies the update workflow in the following way:
\begin{enumerate}
\item $R_{clean} = R_{clean} \cup R_{sat}$
\item $R_{dirty} = R_{viol}$
\item Apply the algorithm in Section \ref{update-alg}.
\end{enumerate}

\vspace{0.5em}

\begin{example}[Rule-Based Detection]\label{detex1}
An example of a rule on the running example dataset is that the \texttt{status} of
a contribution can be only ``allowed" or ``disallowed".
Any other value for \texttt{status} is considered violation.
\end{example}

\subsection{Adaptive Detection}
Rule-based detection is not possible in all cases, especially in cases where the analyst selectively modifies data.
This is why we propose an alternative called the adaptive detector.
Essentially, we reduce the problem to training a classifier on previously cleaned data.
Note that this ``learning" is distinct from the ``learning" in the user-specified statistical model.
One challenge is that detector needs to describe how the data is dirty.
The detector achieves this by categorizing the corruption into $u$ classes, and using a multi-class classifier.
These classes are corruption categories that do not necessarily align with features, but every record is classified with at most one category.

When using adaptive detection, the repair step has to clean the data and report to which of the $u$ classes the corrupted record belongs.
When an example $(x,y)$ is cleaned, the repair step labels it with one of the ${\text{clean}, 1,2,...,u+1}$ classes (including one for ``not dirty").
It is possible that $u$ increases each iteration as more types of dirtiness are discovered.
In many real world datasets, data errors have locality, where similar records tend to be similarly corrupted.
There are usually a small number of error classes even if a large number of records are corrupted.
This problem can be addressed by any classifier, and we use an all-versus-one Logistic Regression in our experiments.

The adaptive detector modifies the update workflow in the following way:
\begin{enumerate}
\item Let $R_{clean}$ be the previously cleaned data, and let $U_{clean}$ be a set of labels for each record indicating the error class and if they are dirty or ``not dirty''.
\item Train a classifier to predict the label \textsf{Train}$(R_{clean}, U_{clean})$
\item Apply the classifier to the dirty data \textsf{Predict}$(R_{dirty})$
\item For all records predicted to be clean, remove from $R_{dirty}$ and append to $R_{clean}$.
\item Apply the algorithm in Section \ref{update-alg}.
\end{enumerate}

The precision and recall of this classifier should be tuned to favor classifying a record as dirty to avoid falsely moving a dirty record into $R_{clean}$. In our experiments, we set this value to $0.90$ probability of the ``not dirty" class.

\vspace{0.75em}

\begin{example}[Adaptive Detection With OpenRefine]\label{detex2}
OpenRefine is a spreadsheet-based tool that allows users to explore and transform data.
However, it is limited to cleaning data that can fit in memory on a single computer.
Since the cleaning operations are coupled with data exploration, \sys does not know what is dirty in advance (the analyst may discover new errors as she cleans).

Suppose the analyst wants to use OpenRefine to clean the running example dataset with \sys.
She takes a sample of data from the entire dataset and uses the tool to discover errors.
For example, she finds that some drugs are incorrectly classified as both drugs and devices.
She then removes the device attribute for all records that have the drug name in question.
As she fixes the records, she tags each one with a category tag of which corruption it belongs to.
\end{example}






\subsection{The Estimator}\label{sampling}
The goal of the estimator is to estimate the gradient of a record w.r.t to the clean data $\nabla\phi(x^{(c)}_i,y^{(c)}_i,\theta^{(t)})$.
\sys leverages previously cleaned data as well as the detector to compute this estimate.
There are a number of different approaches, such as regression, that could be used to estimate the cleaned value given the dirty values.
However, there is a problem of scarcity, where errors may affect a small number of records.
As a result, the regression approach would have to learn a multivariate function with only a few examples.
Consequently, high-dimensional regression is ill-suited for the estimator.
Instead, the estimator uses a linearization of the gradient and average feature-by-feature changes.

\subsection{Estimation Algorithm}
If most of the features are correct, it would seem like the gradient is only
incorrect in one or two of its components.
The problem is that the gradient $\nabla\phi(\cdot)$ can be a very non-linear function of the features that couple features together.
For example, the gradient for linear regression is:
\[
\nabla\phi(x,y,\theta) = (\theta^Tx - y)x
\]
It is not possible to isolate the effect of a change of one feature on the gradient.
Even if one of the features is corrupted, all of the gradient components will be incorrect.

To address this problem, the gradient can be approximated in a way that the effects of dirty features on the gradient are decoupled.
Recall, in the detection problem, that associated with each $r \in R_{dirty}$ are two sets of errors $f_r$ , $l_r$ which identifies the set of corrupted features and labels.
This property can be used to construct a coarse estimate of the clean value.
The main idea is to calculate average changes for each feature, then given an uncleaned (but dirty) record, add these average changes to correct the gradient.

The needed approximation represents a linearization of the errors, and the resulting approximation will be of the form:
\[
p(r)\propto\|\nabla\phi(x,y,\theta^{(t)}) + M_x \cdot \Delta_{rx} +  M_y \cdot \Delta_{ry}\|
\]
where $M_x$, $M_y$ are matrices and $\Delta_{rx}$ and $\Delta_{ry}$ are vectors with one component for each feature and label where each value is the average change for those features that are corrupted and 0 otherwise.
Essentially, if the gradient with respect to the dirty data plus some linear correction factor.
In the technical report, we present a derivation using a Taylor series expansion and a number of $M_x$ and $M_y$ matrices for common convex losses~\cite{activecleanarxiv}.
It also describes how to maintain $\Delta_{rx}$ and $\Delta_{ry}$ as cleaning progresses~\cite{activecleanarxiv}.

\subsubsection{More Accurate Early Error Estimates}\label{acc}
Linearization over avoids amplifying estimation error for small samples.
Consider the linear regression gradient:
\[
\nabla\phi(x,y,\theta) = (\theta^Tx - y)x
\]
This can be rewritten as a vector in each component:
\[
g[i] = \sum_{i} x[i]^2-x[i]y + \sum_{j \ne i} \theta[j]x[j]
\]
This function is already mostly linear in $x$ except for the one quadratic term.
However, this one quadratic term has potential to amplify errors.
Consider two expressions:
\[
f(x+\epsilon) = (x+\epsilon)^2 = x^2 + 2x\epsilon + \epsilon^2
\]
\[
f(x+\epsilon) \approx f(x) + f'(x)(\epsilon) = x^2 + 2x\epsilon
\]
The only difference between the two estimates is the quadratic $\epsilon^2$, if $\epsilon$ is highly uncertain random variable then the quadratic dominates.
If the variance is large, the Taylor estimate avoids amplifying the error.


\iffalse
\subsection{Estimation For Adaptive Case}
A similar procedure holds in the adaptive setting, however, it requires reformulation.
Here, \sys uses $u$ corruption classes provided by the detector.
Instead of conditioning on the features that are corrupted, the estimator conditions on the classes.
So for each error class, it computes a $\Delta_{ux}$ and $\Delta_{uy}$.
These are the average change in the features given that class and the average change in labels given that class.
\[
p(r_u)\propto\|\nabla\phi(x,y,\theta^{(t)}) + M_x \cdot \Delta_{ux} +  M_y \cdot \Delta_{uy}\|
\] 

Here is an example of using the optimization to select a sample of data for cleaning.
\begin{example}\label{estex}
Consider using \sys with an a priori detector.
Let us assume that there are no errors in the labels and only errors in the features.
Then, each training example will have a set of corrupted features (e.g., $\{1,2,6\}$, $\{1,2,15\}$).
Suppose that the cleaner has just cleaned the records $r_1$ and $r_2$ represented as tuples with their corrupted feature set: ($r_1$,$\{1,2,3\}$), ($r_2$,$\{1,2,6\}$).
For each feature $i$, \sys maintains the average change between dirty and clean in a value in a vector $\Delta_x[i]$ for those records corrupted on that feature. 

Then, given a new record ($r_3$,$\{1,2,3,6\}$), $\Delta_{r_3x}$ is the vector $\Delta_x$ where component $i$ is set to 0 if the feature is not corrupted.
Suppose the data analyst is using an SVM, then the $M_x$ matrix is as follows:
\[
M_x[i,i] = \begin{cases}      
-y[i] ~~~~~~\text{ if } y\boldsymbol{x}\cdot\theta < 1 \\
0\ ~~~~~~~\text{ if } y\boldsymbol{x}\cdot\theta \geq 1      
\end{cases} 
\]
Thus, we calculate a sampling weight for record $r_3$:
\[
p(r_3) \propto\|\nabla\phi(x,y,\theta^{(t)}) + M_x \cdot \Delta_{r_3x} \|
\] 
To turn the result into a probability distribution, \sys normalizes over all dirty records.
\end{example}
\fi



\section{Sampling}\label{dist-samp}
In the previous section, we assumed that our model update received a sample with probabilities $p(r)$.
In this section, we derive an optimal sampling problem that directly follows from our analysis of our update rule via SGD.
It will turn out that the solution to the optimal sampling problem is not realizable in practice (as it depends on knowing the cleaned value), but we can use this to inform the next section where we estimate the cleaned value.

\subsection{Goals and Challenges}
In the Machine Learning and Optimization literature, SGD algorithms are optimized to avoid scanning the entire data.
Uniform sampling is cheap so it is the preferred solution.
However, data cleaning costs can be many orders of magnitude higher than model training.
As a result, uniform sampling may not be the most efficient option.
We can sacrifice computational overhead by precomputing some results over the entire data for savings during the data cleaning phase.
We formulate this problem as an optimal sampling problem where we want to compute the sampling probabilities $p(r)$ that maximize the accuracy of our updates.

\subsection{Optimal Sampling Problem}
Recall that the convergence rate of an SGD algorithm is bounded by $\sigma^2$ which is the variance of the gradient.
Intuitively, the variance measures how accurately we estimate the gradient from a uniform sample.
Other sampling distributions, while preserving the sample expected value, may have a lower variance.
Thus, we define the optimal sampling problem as a search over sampling distributions to find the minimum variance sampling distribution.

\begin{definition}[Optimal Sampling Problem]
Given a set of candidate dirty data $R_{dirty}$, $\forall r \in R_{dirty}$ find sampling probabilities $p(r)$ such that over all samples $S$ of size $k$ it minimizes:
\[
\mathbb{E}(\|g_S - g^*\|^2)
\]
\end{definition}
In other words, we want to most accurately (in the mean squared error sense) estimate the gradient. 
To construct these sampling probabilities, we first need the following lemma about importance sampling.
This lemma describes the optimal distribution over a set of scalars:
\begin{lemma}\label{impsample}
Given a set of real numbers $A = \{a_1,...,a_n\}$, let $\hat{A}$ be 
a sample with replacement of $A$ of size k.
If $\mu$ is the mean $\hat{A}$, the sampling distribution that minimizes
 the variance of $\mu$, i.e., the expected square error, is $p(a_i) \propto a_i$.
\end{lemma}
\begin{proof}[Sketch]
This proof follows from \cite{mcbook}, as it is a straight-forward importance sampling result.
We include the proof in the appendix (Section \ref{impsample-deriv})
\end{proof}

Lemma \ref{impsample} shows that when estimating a mean of numbers with sampling, the distribution with optimal variance is where we sample proportionally to the values.
This insight leads to a direct higher-dimensional generalization, where at iteration $t$ we should sample the records in $R_{dirty}$ with probabilities:
\[
p_i \propto \|\nabla\phi(x^{(c)}_i,y^{(c)}_i,\theta^{(t)})\| \blacksquare
\]

However, in our case, it leads to a chicken-and-egg problem.
The optimal sampling distribution requires knowing $(x^{(c)}_i,y^{(c)}_i)$, however, we have to sample and clean those points to get those values.
In the next section, we discuss how to inexpensively approximate this optimal distribution.
As our technique can work with \emph{any} distribution, we are guaranteed convergence no matter how inaccurate this approximation is.
However, a better approximation will lead to an improved convergence rate.

\section{Approximating the Optimal Distribution}
This section describes our solution to problem of constructing and maintaining an inexpensive estimate of clean value $e(\cdot)$.
To do this accurately, is a challenging regression problem.
However, we only use this estimate to construct a sampling distribution, so a coarse-grained estimate suffices.
We show how we can exploit the structure of the data cleaning operations to find this coarse-grained estimate.

\subsection{Error Decoupling}
Recall, that when we formalized the error detection problem, we ensured that associated with each $r \in R_{dirty}$ is a set of errors $e_r$ which is a set that identifies a set of corrupted columns.
Since there is a one-to-many relationship between columns and features, each erroneous column affects a disjoint set of features.
We will show how we can construct a decoupled estimate, one where we independently estimate for each colum and aggregate them together.

Property represents a linearization of the errors, and can be addressed with a first order approximation of the expected gradient.
The expected gradient is defined as:
\[
\mathbb{E}(\nabla\phi(\theta_{(t)}^Tx_{clean},y_{clean}))
\]
We can take the expected value of the Taylor series expansion around the dirty value.
If $d$ is the dirty value and $c$ is the clean value, the Taylor series approximation for a function $f$ is given as follows:
\[
f(c) = f(d) + f'(d)\cdot(d-c) + ...
\]
If we ignore the higher order terms, we see that the linear term $f'(d)\cdot(c-d)$ decouples the features.
We only have know the change in each feature to estimate the change in value.

In our case the function $f$ is actually the gradient $\nabla\phi$.
So, the resulting linearization is:
\[
\nabla\phi(\theta^Tx_{clean},y_{clean}) \approx \nabla\phi(\theta^Tx,y) + \frac{\partial}{\partial X}\nabla\phi(\theta^Tx,y)\cdot (x - x_{clean}) \]
\[+ \frac{\partial}{\partial Y}\nabla\phi(\theta^Tx,y)\cdot (y - y_{clean}) + ...
\]
When we take the expected value:
\[
\mathbb{E}(\nabla\phi(\theta^Tx_{clean},y_{clean})) \approx \nabla\phi(\theta^Tx,y) + \frac{\partial}{\partial X}\nabla\phi(\theta^Tx,y)\cdot \mathbb{E}(\Delta x) \]
\[+ \frac{\partial}{\partial Y}\nabla\phi(\theta^Tx,y)\cdot \mathbb{E}(\Delta y) + ...
\]

\subsection{Maintaining Decoupled Averages}
This linearization allows us to maintain per feature (or label) average changes and use these changes to center the optimal sampling distribution around the expected clean value.
\begin{lemma}[Single Feature]
For a feature $i$, we average all records cleaned that have an error for that feature, weighted by their sampling probability:
\[
\bar{\Delta}_i = \frac{1}{K}\sum_{j=0}^K (x[i]-x_{clean}[i])\times p_j
\]
Similarly, for a label $i$:
\[
\bar{\Delta}_i = \frac{1}{K}\sum_{j=0}^K (y[i]-y_{clean}[i])\times p_j
\]
\end{lemma}

Then, it follows, that we can aggregate the $\bar{\Delta}_i$ into a single vector:
\begin{theorem}[Delta vector]
Let $\{1..,i,...,d\}$ index the set of features and labels.
For a record $r$, the set of corrupted features is $f_r$.
Then, each record $r$ has a d-dimensional vector $\Delta_r$ which is constructed as follows:
\[
 \Delta_r[i] = \begin{cases} 0 & i \notin f_r \\ 
\bar{\Delta}_i & i \in f_r
\end{cases} 
\]
\end{theorem}

With the above theorem, we address the \textbf{Sampling Distribution and Impact Estimate (Problem \ref{imp-samp} and \ref{imp-est}): }
\[p_{r}\propto\|\nabla\phi(x,y,\theta^{(t)}) + \frac{\partial}{\partial X}\nabla\phi \cdot \Delta_r +  \frac{\partial}{\partial Y}\nabla\phi \cdot \Delta_r\|\]

\subsection{Algorithm}
Now that we know how to feedback the error estimates $c(\cdot)$, we describe the entire workflow of \sys:
\begin{enumerate}[noitemsep]
\item Initialize with $\theta^{(0)}$ as the dirty model, $T$ iterations, with a batch size $B$
\item Initialize all $\Delta = 0$
\item For rounds i=1...T
\begin{enumerate}
	\item Sample $B$ candidate dirty data points with probabilites as described.
	\item Apply data cleaning to the sample of data.
	\item Apply weighted gradient descent to update the model.
	\item Update $\Delta$ for each feature.
\end{enumerate}
\item Return $\theta^{(T)}$
\end{enumerate}

\subsection{Physical Design Considerations}
From a systems perspective, the important step in this algorithm is step 3a.
We have to query a sample of candidate data points from $R_{dirty}$ which 
can be expensive.
For eample, if we have a set of constraints, we would have to re-evaluate 
the violated constraints at each iterations.
This is why we make assumptions about the persistence of data repairs in our
problem formalization.
If we make this assumption, we can run the query once at initialization and index 
the dirty records.
Then, as we clean data, we can maintain the index.

\subsection{Comparative Analysis}\label{analysis}
\begin{table*}[ht!]\footnotesize
\centering
\begin{tabular}{ l l l l l}
  Technique & Systematic Errors & Preferred Regime & Cost per Iteration & Overall Cleaning Cost \\ \hline
  Robust Machine Learning & Incomplete & High-Magnitude Random Outliers & - & None \\
  SampleClean & Yes & Very Biased & - & Sample \\
  ActiveLearning & Yes & Minimal Error & O(N) & Reduced\\
  \sys+Uniform & Yes & Sparse, high-variance errors & O(N) & Reduced\\
  \sys+Importance & Yes & Sparse, localized errors & O(N) & Greatly Reduced\\
\end{tabular}
\end{table*}
There is well-studied history of Machine Learning and Data Cleaning with budgets and prioritization. 
Suppose, we have a budget of cleaning $k\ll N$ records.
Non-iterative techniques include the traditionally studied robust Machine Learning.
These techniques handle only a small set of systematic error.
If we want to handle a larger space of data systematic error, then we need some sort of a data cleaning approach.
We could apply SampleClean \cite{wang1999sample} in a non-iterative fashion which takes a random sample of data, applies data cleaning, and trains the model to optimum on the sample.

An alternative iterative technique is Active Learning to prioritize which data to clean.
Prior work in Active Learning is agnostic to data error so the prioritization will be with respect to the dirty data instead of the clean data.
Active Learning may give us a benefit when the statistics of the dirty data are a good proxy for the clean data.
In fact, expected gradient length as an Active Learning criterion has been studied before \cite{settles2010active}, but in this work we devise a technique to compensate for data error.
We also leverage the structure of the data cleaning setting to make more progress at each iteration than a typical Active Learning workflow by averaging the gradients of newly cleaned batches of data with data that we know is clean.
We summarize the salient features in Table(\ref{estimators}).

\reminder{SK will flesh out analysis here next week.}

\reminder{Beats SampleClean in ideal case}
\begin{lemma}
In the oracular case, where we know the clean data in advance, importance sampling gives a strictly tighter error bound than uniform sampling.
\end{lemma}

\reminder{Beats Active Learning in ideal case}
\begin{lemma}
In the oracular case, where we know the clean data in advance, the gradient step made by \sys gives a strictly lower error bound than a naive random sample.
\end{lemma}

\reminder{Gains are preserved in practice}
\begin{lemma}
In the real case, where we have to estimate the clean data, importance sampling gives a tighter error bound when the variance of the estimates $e$ is less than the variance of the gradients.
\end{lemma}





%\section{Progressive Data Cleaning}
In the previous section, we described our algorithmic insight into this problem.
The algorithm treats the data cleaning as a black-box that materializes the clean value,
however, to make such a system work there are numerous data cleaning considerations.

\subsection{Incremental Application of Cleaning}
There are some cases in which cleaning a row of data may require considering the other data in the sample.
Consider entity resolution, where we have to resolve inconsistent representations of an attribute.
When cleaning a sample of data this requires considering all of the inconsistent representations in the sample, and then resolving them to a canonical representation.
However, such operations will fail in the iterative batch mode proposed by \sys.

To avoid this issue, we define the data cleaning operator semantics in the following way.
The data cleaning operation is applied to all previously cleaned points.
At each iteration \sys samples more data, and we have to define our data cleaning operations to incrementally update to the larger sample (including previously cleaned data).
For the entity resolution operator, we define incremental maintenance procedures and fall-back to recomputation if there are user-defined operations.

\noindent \textbf{Entity Resolution Operator: } The entity resolution operator can be incrementally maintained by joining the new subsample with cleaned sample and taking the transitive closure. This can be implemented efficiently using a UNION-FIND data structure.

\subsection{Incremental Re-train}
As we increase the sample size, the already cleaned rows might require updating as well. 
Consider the entity resolution operator, which might find a better canonical representation 
in the new sample.

\reminder{TODO}

\subsection{Physical Design}
To make incremental cleaning feasible, there are also indexing/partitioning considerations.
We need to quickly determine which rows are dirty and which have been previously cleaned.

\reminder{TODO}

\subsection{Choosing Error-Predicate vs. General Case}
Based on the cleaning operator, we can decide whether the operation is in the Error-Predicate Case or General Case.
If we are in the error-predicate case, then we can build a predicate index to determine which records are clean and dirty efficiently. 
\reminder{TODO}

\section{Experiments}
There are a number of different axes on which we can evaluate \sys.
First, we take real datasets and generate various types of errors to illustrate the value of data cleaning in comparison to robust statistical techniques.
Next, we explore different prioritization and model update schemes for data cleaning samples.
Finally, we evaluate \sys end-to-end in a number of real-world data cleaning scenarios.

\subsection{Experimental Setup and Notation}
Every experiment has two steps: data cleaning and model evaluation.
We evaluate the data cleaning on one metric:

\noindent\textbf{Cleaning Efficiency. } Let $K$ be the number of samples processed by the algorithm, and $K'$ be the number of samples that were actually dirty. The cleaning efficiency is $\frac{K'}{K}$.

In our experiments, we explore three classification models: L1-Hinge Loss SVM, Logistic Regression, and Thresholded Linear Regression.
We evaluate the trained models on the following metrics:

\noindent\textbf{Relative Model Error. } Let $\theta$ be the model trained on the dirty data, and let $\theta^*$ be the model trained on the same data if it was cleaned. Then the model error is defined as $\frac{\|\theta - \theta^*\|}{\|\theta^*\|}$.

\noindent\textbf{Testing Accuracy. } Let $\theta$ be the model trained on the dirty data, and let $\theta^*$ be the model trained on the same data if it was cleaned. Let $T(\theta)$ be the out-of-sample testing accuracy when the dirty model is applied to the clean data, and $T(\theta^*)$ be the testing accuracy when the clean model is applied to the clean data. The testing error is defined as $T(\theta^*) - T(\theta)$

\subsubsection{Scenarios}
We apply these models in the following scenarios:

\noindent\textbf{Housing: } In this dataset, our task is to predict housing prices from 13 numerical and categorical covariates. There are 550 data points in this dataset. The model is a Logistic Regression classifier which predicts if the house price is greater than \$500k.

\noindent\textbf{Adult: } In this census dataset, our task is to predict the income bracket (binary) from 12 numerical and categorical covariates. There are 45552 data points in this dataset. We use a SVM classifier to predict the income bracket of the person.

\noindent\textbf{EEG: } In this dataset, our task is to predict the on set of a seizure (binary) from 15 numerical covariates. There are 14980 data points in this dataset. This dataset is unique because the classification is hard with linear predictors. The model that we use is a thresholded Linear Regression.

\noindent\textbf{MNIST: } In this dataset, our task is to classify 60,000 images of handwritten images into 10 categories. The unique part of this dataset is the featurized data consists of a 784 dimensional vector which includes edge detectors and raw image patches. We use this dataset to explore how we can corrupt the raw data to affect subsequent featurization. The model is an one-to-all multiclass SVM classifier. 

\subsection{Experiment 1. Effect of Cleaning}
Before we evaluate \sys, we first evaluate the benefits of cleaning on our 4 example datasets.
We first explore this problem without sampling to understand which types of errors are amenable to data cleaning and which are better suited for robust statistical techniques.
We compare 4 schemes: (1) cleaning, (2) adding an L2 regularizer tuned to maximal accuracy with a grid search, (3) discarding the dirty data, and (4) baseline of no cleaning.

We corrupted 5\% of the training examples in each dataset.
We corrupted these data in two different ways.

\noindent\textbf{Random errors: } We simulated high-magnitude random outliers. We select 5\% of the examples and features uniformly at random and replace a feature with 3 times the highest feature value.

\noindent\textbf{Systematic errors: } We simulated innocuous looking (but still incorrect) systematic errors. We trained the model on the clean data, find the most important feature (highest weighted). We sort examples but this feature and corrupt the top 5\% of examples with the mean value for that feature.

\begin{figure}[ht!]
\centering
 \includegraphics[width=0.8\columnwidth]{exp/exp2.pdf}
 \includegraphics[width=0.8\columnwidth]{exp/exp1.pdf}
 \caption{Robust techniques work best when corrupted data are random and look atypical. Data cleaning can provide reliable performance in both the systematically corrupted setting and randomly corrupted setting.\label{sys-rand}}
\end{figure}

In Figure \ref{sys-rand}, we present the results of this experiment.
As we argued in this paper, the robust method performs well on the random high-magnitude outliers, however, falters on the systematic corruption.
Interestingly enough, in the random setting, discarding dirty data also performs well.
However, when errors are systematic data cleaning is the most reliable option across datasets.
In the MNIST dataset, we see a particularly significant effect of systematic corruption
where the test accuracy drops from nearly 98\% to 78\%.
Multiclass classification is particularly sensitive to systematic corruption when the corruptions can make classes ambiguous (e.g. reconizing a ``4" and a ``9").
The problem is that a priori, we do not know if data error is random or systematic.
While data cleaning requires more effort, it provides benefits in both settings.

\subsection{Experiment 2. Prioritization}
The next set of experiments evaluate different approaches to cleaning a sample of data.
In this set of experiments, we use the random errors generated above.

\subsubsection{2a. Alternative Algorithms}
In our first prioritization experiment, we evaluate the samples-to-error tradeoff between three alternative algorithms:

\noindent\textbf{SampleClean (SC): } In SampleClean, we do not use a gradient update and instead take a sample of data and train the model to completion on the sample.

\noindent\textbf{Active Learning (AL): } In Active Learning, we do not consider the effect of ``data cleaning" and prioritze points by their dirty gradient value. We do, however, do this iteratively and update the model.

\noindent\textbf{ActiveClean Oracle (AC+O): } In ActiveClean Oracle, we importance sample points by their clean gradient. This represents the theoretical best that our algorithm could hope to achieve given perfect error estimation.

In Figure \ref{prio-perf}, we present our results on Housing, Adult, and EEG. 
We find that \sys gives its largest benefits for small sample sizes (up-to 12x).
\sys makes significant progress because of its intelligent initialization, iterative updates, and partitioning.
For example, the EEG dataset is the hardest classification task.
SampleClean has difficulty on this dataset since it takes a uniform sample of data (only 5\% of which are corrupted on average) and tries to train a model using only this data.
\sys and Active Learning leverage the initialization from the dirty data to get an improved result. 
However, \sys's impact estimates and error partitioning allow us to beat Active Learning on all three of the datasets.

\begin{figure*}[t]
\centering
 \includegraphics[scale=0.15]{exp/exp3a.pdf}
 \includegraphics[scale=0.15]{exp/exp3b.pdf}
  \includegraphics[scale=0.15]{exp/exp3c.pdf}
 \caption{\sys converges with a smaller sample size to the true result in comparison to Active Learning and SampleClean. \label{prio-perf}}
\end{figure*}

\subsubsection{2b. Source of Improvements}
Throughout the paper, we proposed numerous optimizations.
Now, we try to understand the source of our improvements w.r.t Active Learning and SampleClean.
We pick a single point on the curves shown in Figure \ref{prio-perf} that corresponds to 10\% of the data cleaned (55 for Housing, 4555 for Adult, 150 for EEG) and compare the performance of \sys with and without various optimizations.
We denote \sys without partitioning as (AC-P) and \sys without partitioning and importance sampling as (AC-P-I).
In Figure \ref{opts}, we plot the relative error of the alternatives w.r.t to the optimized version of \sys.
Partitioning significantly improves our results in all of the datasets, and accounts for a substantial part of the improvements over Active Learning.
However, when we remove partitioning we still see some improvements since our importance sampling relies on error impact estimates that judge how valuable a point is to the clean model rather than the dirty model in Active Learning.
Not surprisingly, when we remove both these optimizations, \sys is comparable or slightly worse than Active Learning.

\begin{figure}[ht!]
\centering
 \includegraphics[width=\columnwidth]{exp/exp8.pdf}
 \caption{We clean 10\% of the data with the alternative algorithms and also include variants of \sys with optimization removed. We plot the relative error w.r.t the optimized \sys. Both partitioning and importance sampling lead to significant reductions in error. \label{opts}}
\end{figure}

We evalue Active Learning and \sys to better understand this relationship.
In Figure \ref{albias}, we vary the biasing effect of our random corruptions.
That is, we start with zero mean noise and increase the mean value and variance of the noise.
Since Active Learning uses the gradient, if there is zero mean noise, in expectation, the dirty data and clean data are the same.
However, as the bias increases, the fact that Active Learning prioritizes w.r.t to the dirty data matters more and becomes increasingly erroneous w.r.t to \sys.

\begin{figure}[ht!]
\centering
 \includegraphics[width=0.6\columnwidth]{exp/exp10.pdf}
 \caption{As we increase the biasing nature of the corruption, Active Learning is increasingly erroneous w.r.t \sys. \label{albias}}
\end{figure}

\subsubsection{2c. Error Dependence}
Both Active Learning and \sys outperform SampleClean in our experiments.
In our next experiment, we try to understand how much of this performance 
is due to the initialization (i.e., SampleClean trains a model from ``scratch").
We vary the rate of random error, thus making the initialization more and more arbitrary, 
and measure the relative performance between SampleClean and \sys.
Since SampleClean only acts on a clean sample of data, it is robust to data error.
So at some point, the errors in the data are so significant that training a model on a small but clean sample of data is more efficient than iteratively updating the dirty model.

In Figure \ref{bias}, we present the results from this experiment.
We corrupt entries from the data matrix of the Adult dataset at random (probability on plotted on the x-axis).
Then, we measure the number of records we need to clean before we have a relative error of 0.1\%.
We find that at about 30\% corruption rate, SampleClean is more accurate than \sys.
Since the Adult dataset has 12 features, a 30\% corruption rate corresponds to each example with 3.6 features incorrect on average.
We optimized \sys for sparse and relatively small errors but it still shows reasonable performance even in this highly erroneous setting. 
At higher corruption rates, \sys requires more than one epoch to converge to an accurate answer which requires cleaning almost all of the data.

\begin{figure}[ht!]
\centering
 \includegraphics[width=0.6\columnwidth]{exp/exp9.pdf}
 \caption{We corrupt an increasing number of entries in the data matrix. At about 30\% corrupted, \sys is no longer more efficient than SampleClean. \label{bias}}
\end{figure}

\subsubsection{2d. Testing Accuracy}
In the previous experiments, we studied the relative model error which measures the training loss. 
However, to an end user the metric that matters is test accuracy.
In the next experiment, we try to understand how reductions in model error correlate to improvements in test error.
In Figure \ref{prio-tperf}, we present the results for the three datasets: Adult, Housing, and EEG.
We find that in two of the datasets, Housing and Adult, \sys converges to clean test accuracy faster than the alternatives.

However, there is a curious negative result with the EEG dataset that we would like to highlight. 
We find that even though \sys has significantly lower model error (Figure \ref{prio-perf}), this does not correspond to as significant of an increase in test accuracy.
We speculate this is due to the inherrent hardness of the EEG classification problem.
\sys may encourage overfitting at intermediate results for hard classification tasks.
The solution to this problem may be to add additional regularization, thus actually changing the optimization problem.
We hope to explore this problem in further detail in future work.

\begin{figure*}[t]
\centering
 \includegraphics[scale=0.15]{exp/exp3aa.pdf}
 \includegraphics[scale=0.15]{exp/exp3bb.pdf}
  \includegraphics[scale=0.15]{exp/exp3cc.pdf}
 \caption{\sys converges with a smaller sample size to the maximum test accuracy in comparison to Active Learning and SampleClean. \label{prio-tperf}}
\end{figure*}

\subsection{Experiment 3. Error Predicates vs. Classification}
In the next set of experiments, we explore the error partitioning in more detail.
We presented two models for error sampling, one where we are given a set of candidate dirty records through a predicate and one where we have to learn this predicate as we clean.

\reminder{Evaluate how much we lose}
\begin{figure}[ht!]
\centering
  \includegraphics[scale=0.15]{exp/exp5c.pdf}
 \caption{\reminder{TODO} \label{tradeoffs2}}
\end{figure}

\subsection{Error Types}
Now, we evaluate the tradeoff space between random errors and systematic errors.
In the first plot (Figure \ref{tradeoffs}a), we show the relative accuracy between AC+U and AC as a function of predicate randomness. 
In other words, we start with the 5\% systematic errors that we had before, and increasingly make them more random (i.e., corrupt a random point with $\epsilon$ probability).
As the errors become more random, AC has less of a benefit in comparison to uniform sampling.
Next, in Figure \ref{tradeoffs}b, we explore the opposite tradeoff. 
We start with random errors and vary the magnitude of corruption.
When errors are random with 0 mean, then AC is equivalent to Active Learning.
As we increase the bias, the impact estimates from AC change the sampling distribution and there it is more accurate in comparison.  
\begin{figure}[ht!]
\centering
 \includegraphics[scale=0.13]{exp/exp5a.pdf}
 \includegraphics[scale=0.13]{exp/exp5b.pdf}
 \caption{\reminder{The labels are wrong and these plots are confusing} \label{tradeoffs}}
\end{figure}



\subsection{Incremental vs. Sample}
\reminder{How much bias before SC becomes useful?}

\subsection{Price of a Scan}
\reminder{With partitioning, without partitioning, with(out) indexing?}

\subsection{Real Scenarios}
\reminder{Value Filling: MNIST}

\begin{figure*}[t]
\centering
 \includegraphics[scale=0.25]{exp/5x5removal.png}
 \includegraphics[scale=0.15]{exp/exp7a.pdf}
 \caption{One experiment with MNIST}
\end{figure*}

\reminder{CFD: Adult}
\section{Related Work}
Bringing together data cleaning and machine learning presents us with several exciting new research opportunities that incorprates results from both communities.
We highlight some of the key relevant work in this field and how this relates to our proposal.

\noindent \textbf{Stochastic Optimization: } Zhao and Tong recently proposed using importance sampling in conjunction with stochastic gradient descent \cite{zhao2014stochastic}. However, calculating the optimal importance sampling distribution is very expensive and is only justified in our case because data cleaning is even more expensive. Zhao and Tong use an approximation to work around this problem. This work is one of many in an emerging conensus in stochastic optimization that not all data are equal (e.g., \cite{qu2014randomized}). This line of work builds on prior results in linear algebra that show that some matrix columns are more informative than others \cite{rineas2012fast}, and Active Learning which shows that some labels are more informative that others \cite{settles2010active}.

\noindent \textbf{Active vs. Transfer Learning: } While it is natural to draw the connection between \sys and Active Learning, which is widely used in data cleaning, they differ in a few crucial ways. 
Active Learning largely studies the problem of label acquisition \cite{settles2010active}.
This can be seen as a narrower problem setting than our problem (missing data in the label attribute), and in fact, our proposed approach can be viewed as an Active Learning algorithm.
\sys has a stronger link to a field called Transfer Learning \cite{pan2010survey}. The basic idea of Transfer Learning is that suppose a model is trained on a dataset $D$ but tested on a dataset $D'$. In transfer learning, the model is often weighted or transformed in such a way that it can still predict on $D'$. Transfer Learning has not considered the data cleaning setting, in which there is a bijective map between $D \mapsto D'$ that is expensive to compute. Much of the complexity and contribution of our work comes from efficiently cleaning the data.

\noindent \textbf{Secure Learning: } Another relevant line of work is the work in private machine learning  \cite{wainwright2012privacy, duchi2013local}. Learning is performed on a noisy variant of the data which mitigates privacy concerns. The goal is to extrapolate the insights from the noisy data to the hidden real data. Our results are applicable in this setting in the following way. Imagine, we were allowed to query $k$ true data points from the real data, which points are the most valuable to query. This is also related work in adversarial learning \cite{nelson2012query}, where the goal is to make models robust to adversarial data manipulation.

\noindent \textbf{Data Cleaning: } There are also several recent results in data cleaning that we would like to highlight. Altowim et al. proposed a framework for progressive entity resolution \cite{altowim2014progressive}. As in our work, this work studies the tradeoff between resolution cost and result accuracy. This work presents many important ideas on which we build in our paper: (1) it recognizes that ER is expensive and some operations are more valuable than others, and (2) anytime behavior is desirable. We take these ideas one step further where we pushdown the model at the end of the pipeline to data cleaning and choose data that is most valuable to the model. Volkovs et al. explored a topic called progressive data cleaning \cite{volkovs2014continuous}. They looked at maintaining constraint-based data cleaning rules as base data changes. Many of the database tricks employed including indexing and incremental maintenance were valuable insights for our work. Bergman et al. explore the problem of query-oriented data cleaning \cite{bergman2015query}. Given a query they clean data relevant to that query. Bergman et al. does not explore aggregates or the Machine Learning models studied in this work.


\section{Conclusion}
In this paper, we propose \sysfull (\sys), an anytime framework for training Machine Learning models with a data cleaning budget.
Naive solutions to this problem can cause sampling errors to dominate any benefit of data cleaning, so instead we propose a gradient-based update to incrementally correct a dirty model.
To make this update as impactful as possible, we exploit many properties we know about data cleaning such as the relative ease of error enumeration.
Our solution is a linear approximation of the optimal sampling distribution which empirically shows significant improvements over alternative approaches.
This formulation fits into a Stochastic Gradient Descent theoretical framework, which allows us to ensure convergence and statistical consistency under relatively mild conditions.
The elegance of the SGD formulation is that we can use approximations of approximations and still have a methodology that gives bounded results.

\sys is only a first step in a larger integration of data analytics and data aquisition/cleaning. 
There are several exciting, new avenues for future work.
First, in this work, we largely study how knowing the Machine Learning model can optimize data cleaning.
We also believe that the reverse is true, knowing the data cleaning operations and the featurization can optimize model training.
For example, applying Entity Resolution to one-hot encoded features results in a linear transformation of the feature space.
For some types of Machine Learning models, we may be able to avoid re-training.
The optimizations described in \sys are not only restricted to SGD algorithms. 
We believe we can extend a variant of SDCA (Stochastic Dual Coordinate Ascent) \cite{jaggi2014communication} to extend this technique to kernelized methods.

%\input{outlier.tex}
%\input{analysis.tex}
%\section{Experiments}
There are a number of different axes on which we can evaluate \sys.
First, we take real datasets and generate various types of errors to illustrate the value of data cleaning in comparison to robust statistical techniques.
Next, we explore different prioritization and model update schemes for data cleaning samples.
Finally, we evaluate \sys end-to-end in a number of real-world data cleaning scenarios.

\subsection{Experimental Setup and Notation}
Every experiment has two steps: data cleaning and model evaluation.
We evaluate the data cleaning on one metric:

\noindent\textbf{Cleaning Efficiency. } Let $K$ be the number of samples processed by the algorithm, and $K'$ be the number of samples that were actually dirty. The cleaning efficiency is $\frac{K'}{K}$.

In our experiments, we explore three classification models: L1-Hinge Loss SVM, Logistic Regression, and Thresholded Linear Regression.
We evaluate the trained models on the following metrics:

\noindent\textbf{Relative Model Error. } Let $\theta$ be the model trained on the dirty data, and let $\theta^*$ be the model trained on the same data if it was cleaned. Then the model error is defined as $\frac{\|\theta - \theta^*\|}{\|\theta^*\|}$.

\noindent\textbf{Testing Accuracy. } Let $\theta$ be the model trained on the dirty data, and let $\theta^*$ be the model trained on the same data if it was cleaned. Let $T(\theta)$ be the out-of-sample testing accuracy when the dirty model is applied to the clean data, and $T(\theta^*)$ be the testing accuracy when the clean model is applied to the clean data. The testing error is defined as $T(\theta^*) - T(\theta)$

\subsubsection{Scenarios}
We apply these models in the following scenarios:

\noindent\textbf{Housing: } In this dataset, our task is to predict housing prices from 13 numerical and categorical covariates. There are 550 data points in this dataset. The model is a Logistic Regression classifier which predicts if the house price is greater than \$500k.

\noindent\textbf{Adult: } In this census dataset, our task is to predict the income bracket (binary) from 12 numerical and categorical covariates. There are 45552 data points in this dataset. We use a SVM classifier to predict the income bracket of the person.

\noindent\textbf{EEG: } In this dataset, our task is to predict the on set of a seizure (binary) from 15 numerical covariates. There are 14980 data points in this dataset. This dataset is unique because the classification is hard with linear predictors. The model that we use is a thresholded Linear Regression.

\noindent\textbf{MNIST: } In this dataset, our task is to classify 60,000 images of handwritten images into 10 categories. The unique part of this dataset is the featurized data consists of a 784 dimensional vector which includes edge detectors and raw image patches. We use this dataset to explore how we can corrupt the raw data to affect subsequent featurization. The model is an one-to-all multiclass SVM classifier. 

\subsection{Experiment 1. Effect of Cleaning}
Before we evaluate \sys, we first evaluate the benefits of cleaning on our 4 example datasets.
We first explore this problem without sampling to understand which types of errors are amenable to data cleaning and which are better suited for robust statistical techniques.
We compare 4 schemes: (1) cleaning, (2) adding an L2 regularizer tuned to maximal accuracy with a grid search, (3) discarding the dirty data, and (4) baseline of no cleaning.

We corrupted 5\% of the training examples in each dataset.
We corrupted these data in two different ways.

\noindent\textbf{Random errors: } We simulated high-magnitude random outliers. We select 5\% of the examples and features uniformly at random and replace a feature with 3 times the highest feature value.

\noindent\textbf{Systematic errors: } We simulated innocuous looking (but still incorrect) systematic errors. We trained the model on the clean data, find the most important feature (highest weighted). We sort examples but this feature and corrupt the top 5\% of examples with the mean value for that feature.

\begin{figure}[ht!]
\centering
 \includegraphics[width=0.8\columnwidth]{exp/exp2.pdf}
 \includegraphics[width=0.8\columnwidth]{exp/exp1.pdf}
 \caption{Robust techniques work best when corrupted data are random and look atypical. Data cleaning can provide reliable performance in both the systematically corrupted setting and randomly corrupted setting.\label{sys-rand}}
\end{figure}

In Figure \ref{sys-rand}, we present the results of this experiment.
As we argued in this paper, the robust method performs well on the random high-magnitude outliers, however, falters on the systematic corruption.
Interestingly enough, in the random setting, discarding dirty data also performs well.
However, when errors are systematic data cleaning is the most reliable option across datasets.
In the MNIST dataset, we see a particularly significant effect of systematic corruption
where the test accuracy drops from nearly 98\% to 78\%.
Multiclass classification is particularly sensitive to systematic corruption when the corruptions can make classes ambiguous (e.g. reconizing a ``4" and a ``9").
The problem is that a priori, we do not know if data error is random or systematic.
While data cleaning requires more effort, it provides benefits in both settings.

\subsection{Experiment 2. Prioritization}
The next set of experiments evaluate different approaches to cleaning a sample of data.
In this set of experiments, we use the random errors generated above.

\subsubsection{2a. Alternative Algorithms}
In our first prioritization experiment, we evaluate the samples-to-error tradeoff between three alternative algorithms:

\noindent\textbf{SampleClean (SC): } In SampleClean, we do not use a gradient update and instead take a sample of data and train the model to completion on the sample.

\noindent\textbf{Active Learning (AL): } In Active Learning, we do not consider the effect of ``data cleaning" and prioritze points by their dirty gradient value. We do, however, do this iteratively and update the model.

\noindent\textbf{ActiveClean Oracle (AC+O): } In ActiveClean Oracle, we importance sample points by their clean gradient. This represents the theoretical best that our algorithm could hope to achieve given perfect error estimation.

In Figure \ref{prio-perf}, we present our results on Housing, Adult, and EEG. 
We find that \sys gives its largest benefits for small sample sizes (up-to 12x).
\sys makes significant progress because of its intelligent initialization, iterative updates, and partitioning.
For example, the EEG dataset is the hardest classification task.
SampleClean has difficulty on this dataset since it takes a uniform sample of data (only 5\% of which are corrupted on average) and tries to train a model using only this data.
\sys and Active Learning leverage the initialization from the dirty data to get an improved result. 
However, \sys's impact estimates and error partitioning allow us to beat Active Learning on all three of the datasets.

\begin{figure*}[t]
\centering
 \includegraphics[scale=0.15]{exp/exp3a.pdf}
 \includegraphics[scale=0.15]{exp/exp3b.pdf}
  \includegraphics[scale=0.15]{exp/exp3c.pdf}
 \caption{\sys converges with a smaller sample size to the true result in comparison to Active Learning and SampleClean. \label{prio-perf}}
\end{figure*}

\subsubsection{2b. Source of Improvements}
Throughout the paper, we proposed numerous optimizations.
Now, we try to understand the source of our improvements w.r.t Active Learning and SampleClean.
We pick a single point on the curves shown in Figure \ref{prio-perf} that corresponds to 10\% of the data cleaned (55 for Housing, 4555 for Adult, 150 for EEG) and compare the performance of \sys with and without various optimizations.
We denote \sys without partitioning as (AC-P) and \sys without partitioning and importance sampling as (AC-P-I).
In Figure \ref{opts}, we plot the relative error of the alternatives w.r.t to the optimized version of \sys.
Partitioning significantly improves our results in all of the datasets, and accounts for a substantial part of the improvements over Active Learning.
However, when we remove partitioning we still see some improvements since our importance sampling relies on error impact estimates that judge how valuable a point is to the clean model rather than the dirty model in Active Learning.
Not surprisingly, when we remove both these optimizations, \sys is comparable or slightly worse than Active Learning.

\begin{figure}[ht!]
\centering
 \includegraphics[width=\columnwidth]{exp/exp8.pdf}
 \caption{We clean 10\% of the data with the alternative algorithms and also include variants of \sys with optimization removed. We plot the relative error w.r.t the optimized \sys. Both partitioning and importance sampling lead to significant reductions in error. \label{opts}}
\end{figure}

We evalue Active Learning and \sys to better understand this relationship.
In Figure \ref{albias}, we vary the biasing effect of our random corruptions.
That is, we start with zero mean noise and increase the mean value and variance of the noise.
Since Active Learning uses the gradient, if there is zero mean noise, in expectation, the dirty data and clean data are the same.
However, as the bias increases, the fact that Active Learning prioritizes w.r.t to the dirty data matters more and becomes increasingly erroneous w.r.t to \sys.

\begin{figure}[ht!]
\centering
 \includegraphics[width=0.6\columnwidth]{exp/exp10.pdf}
 \caption{As we increase the biasing nature of the corruption, Active Learning is increasingly erroneous w.r.t \sys. \label{albias}}
\end{figure}

\subsubsection{2c. Error Dependence}
Both Active Learning and \sys outperform SampleClean in our experiments.
In our next experiment, we try to understand how much of this performance 
is due to the initialization (i.e., SampleClean trains a model from ``scratch").
We vary the rate of random error, thus making the initialization more and more arbitrary, 
and measure the relative performance between SampleClean and \sys.
Since SampleClean only acts on a clean sample of data, it is robust to data error.
So at some point, the errors in the data are so significant that training a model on a small but clean sample of data is more efficient than iteratively updating the dirty model.

In Figure \ref{bias}, we present the results from this experiment.
We corrupt entries from the data matrix of the Adult dataset at random (probability on plotted on the x-axis).
Then, we measure the number of records we need to clean before we have a relative error of 0.1\%.
We find that at about 30\% corruption rate, SampleClean is more accurate than \sys.
Since the Adult dataset has 12 features, a 30\% corruption rate corresponds to each example with 3.6 features incorrect on average.
We optimized \sys for sparse and relatively small errors but it still shows reasonable performance even in this highly erroneous setting. 
At higher corruption rates, \sys requires more than one epoch to converge to an accurate answer which requires cleaning almost all of the data.

\begin{figure}[ht!]
\centering
 \includegraphics[width=0.6\columnwidth]{exp/exp9.pdf}
 \caption{We corrupt an increasing number of entries in the data matrix. At about 30\% corrupted, \sys is no longer more efficient than SampleClean. \label{bias}}
\end{figure}

\subsubsection{2d. Testing Accuracy}
In the previous experiments, we studied the relative model error which measures the training loss. 
However, to an end user the metric that matters is test accuracy.
In the next experiment, we try to understand how reductions in model error correlate to improvements in test error.
In Figure \ref{prio-tperf}, we present the results for the three datasets: Adult, Housing, and EEG.
We find that in two of the datasets, Housing and Adult, \sys converges to clean test accuracy faster than the alternatives.

However, there is a curious negative result with the EEG dataset that we would like to highlight. 
We find that even though \sys has significantly lower model error (Figure \ref{prio-perf}), this does not correspond to as significant of an increase in test accuracy.
We speculate this is due to the inherrent hardness of the EEG classification problem.
\sys may encourage overfitting at intermediate results for hard classification tasks.
The solution to this problem may be to add additional regularization, thus actually changing the optimization problem.
We hope to explore this problem in further detail in future work.

\begin{figure*}[t]
\centering
 \includegraphics[scale=0.15]{exp/exp3aa.pdf}
 \includegraphics[scale=0.15]{exp/exp3bb.pdf}
  \includegraphics[scale=0.15]{exp/exp3cc.pdf}
 \caption{\sys converges with a smaller sample size to the maximum test accuracy in comparison to Active Learning and SampleClean. \label{prio-tperf}}
\end{figure*}

\subsection{Experiment 3. Error Predicates vs. Classification}
In the next set of experiments, we explore the error partitioning in more detail.
We presented two models for error sampling, one where we are given a set of candidate dirty records through a predicate and one where we have to learn this predicate as we clean.

\reminder{Evaluate how much we lose}
\begin{figure}[ht!]
\centering
  \includegraphics[scale=0.15]{exp/exp5c.pdf}
 \caption{\reminder{TODO} \label{tradeoffs2}}
\end{figure}

\subsection{Error Types}
Now, we evaluate the tradeoff space between random errors and systematic errors.
In the first plot (Figure \ref{tradeoffs}a), we show the relative accuracy between AC+U and AC as a function of predicate randomness. 
In other words, we start with the 5\% systematic errors that we had before, and increasingly make them more random (i.e., corrupt a random point with $\epsilon$ probability).
As the errors become more random, AC has less of a benefit in comparison to uniform sampling.
Next, in Figure \ref{tradeoffs}b, we explore the opposite tradeoff. 
We start with random errors and vary the magnitude of corruption.
When errors are random with 0 mean, then AC is equivalent to Active Learning.
As we increase the bias, the impact estimates from AC change the sampling distribution and there it is more accurate in comparison.  
\begin{figure}[ht!]
\centering
 \includegraphics[scale=0.13]{exp/exp5a.pdf}
 \includegraphics[scale=0.13]{exp/exp5b.pdf}
 \caption{\reminder{The labels are wrong and these plots are confusing} \label{tradeoffs}}
\end{figure}



\subsection{Incremental vs. Sample}
\reminder{How much bias before SC becomes useful?}

\subsection{Price of a Scan}
\reminder{With partitioning, without partitioning, with(out) indexing?}

\subsection{Real Scenarios}
\reminder{Value Filling: MNIST}

\begin{figure*}[t]
\centering
 \includegraphics[scale=0.25]{exp/5x5removal.png}
 \includegraphics[scale=0.15]{exp/exp7a.pdf}
 \caption{One experiment with MNIST}
\end{figure*}

\reminder{CFD: Adult}
%\section{Conclusion}
In this paper, we propose \sysfull (\sys), an anytime framework for training Machine Learning models with a data cleaning budget.
Naive solutions to this problem can cause sampling errors to dominate any benefit of data cleaning, so instead we propose a gradient-based update to incrementally correct a dirty model.
To make this update as impactful as possible, we exploit many properties we know about data cleaning such as the relative ease of error enumeration.
Our solution is a linear approximation of the optimal sampling distribution which empirically shows significant improvements over alternative approaches.
This formulation fits into a Stochastic Gradient Descent theoretical framework, which allows us to ensure convergence and statistical consistency under relatively mild conditions.
The elegance of the SGD formulation is that we can use approximations of approximations and still have a methodology that gives bounded results.

\sys is only a first step in a larger integration of data analytics and data aquisition/cleaning. 
There are several exciting, new avenues for future work.
First, in this work, we largely study how knowing the Machine Learning model can optimize data cleaning.
We also believe that the reverse is true, knowing the data cleaning operations and the featurization can optimize model training.
For example, applying Entity Resolution to one-hot encoded features results in a linear transformation of the feature space.
For some types of Machine Learning models, we may be able to avoid re-training.
The optimizations described in \sys are not only restricted to SGD algorithms. 
We believe we can extend a variant of SDCA (Stochastic Dual Coordinate Ascent) \cite{jaggi2014communication} to extend this technique to kernelized methods.
\vspace{-1em}

%\bibliographystyle{abbrv}
%\scriptsize
\fontsize{8.2pt}{8.5pt} \selectfont
\bibliographystyle{abbrv}
\bibliography{ref} 
\clearpage
\normalsize \selectfont
\section{Appendix}
\subsection{Proof of Lemma \ref{unbiased1}}\label{unbiased1-deriv}
\begin{lemma}
The gradient estimate $g(\theta)$ is unbiased if $g_S$ is an unbiased estimate of:
\[
\frac{1}{\mid R_{dirty} \mid} \sum g_i(\theta)
\]
\end{lemma}
\begin{proof}[Sketch]
\[
\mathbb{E}(\frac{1}{\mid R_{dirty} \mid} \sum g_i(\theta)) = \frac{1}{\mid R_{dirty} \mid} \cdot \mathbb{E}(\sum g_i(\theta)))
\]
By symmetry, 
\[
\mathbb{E}(\frac{1}{\mid R_{dirty} \mid} \sum g_i(\theta)) = g(\theta)
\]
\[
\mathbb{E}(\frac{1}{\mid R_{dirty} \mid} \sum g_i(\theta)) = \frac{\mid R_{dirty} \mid \cdot g_S + \mid R_{clean} \mid \cdot g_C  }{\mid R \mid}
\]
\end{proof}

\subsection{Proof of Lemma \ref{impsample}}\label{impsample-deriv}
The variance of this estimate is given by:
\[
Var(\mu) = \mathbb{E}(\mu^2)-\mathbb{E}(\mu)^2
\] 
Since the estimate is unbiased, we can replace $\mathbb{E}(\mu)$ with the average of $A$:
\[
Var(\mu) = \mathbb{E}(\mu^2)-\bar{A}^2
\]
Since $\bar{A}$ is deterministic, we can remove that term during minimization.
Furthermore, we can write $\mathbb{E}(\mu^2)$ as:
\[
\mathbb{E}(\mu^2) = \frac{1}{n^2}\sum_i^n \frac{a_i^2}{p_i}
\]
Then, we can solve the following optimization problem (removing the proportionality of $\frac{1}{n^2}$) over the set of weights $P=\{p(a_i)\}$:
\[
\min_{P} \sum_i^N \frac{a_i^2}{p_i}
\]
\[
\text{subject to: } P > 0, \sum P = 1
\]
Applying Lagrange multipliers, an equivalent unconstrained optimization problem is:
\[
\min_{P > 0,\lambda > 0} \sum_i^N \frac{a_i^2}{p_i} + \lambda \cdot (\sum P - 1)
\]
If, we take the derivatives with respect to $p_i$ and set them equal to zero:
\[
-\frac{a_i^2}{2 \cdot p_i^2} + \lambda = 0
\]
If, we take the derivative with respect to $\lambda$ and set it equal to zero:
\[
\sum P - 1
\]
Solving the system of equations, we get:
\[
p_i = \frac{\mid a_i \mid }{\sum_i \mid a_i \mid}
\]

\subsection{Non-convex losses}\label{non-convex}
We acknowledge that there is an increasing popularity of non-convex losses in the Neural Network and Deep Learning literature. 
However, even for these losses, gradient descent techniques still apply. 
Instead of converging to a global optimum they converge to a locally optimal value. 
Likewise, \sys will converge to the closest locally optimal value to the dirty model. 
Because of this, it is harder to reason about the results.
Different initializations will lead to different local optima, and thus, introduces a complex dependence on the initialization with the dirty model.
This problem is not fundemental to \sys and any gradient technique suffers this challenge for general non-convex losses, and we hope to explore this more in the future.

\subsection{Taylor Approximation}\label{taylor-deriv}
We can take the expected value of the Taylor series expansion around the dirty value.
If $d$ is the dirty value and $c$ is the clean value, the Taylor series approximation for a function $f$ is given as follows:
\[
f(c) = f(d) + f'(d)\cdot(d-c) + ...
\]
If we ignore the higher order terms, we see that the linear term $f'(d)\cdot(d-c)$ decouples the features.
We only have to know the change in each feature to estimate the change in value.
In our case the function $f$ is the gradient $\nabla\phi$.
So, the resulting linearization is:
\[
\nabla\phi(x^{(clean)}_i,y^{(clean)}_i,\theta) \approx \nabla\phi(x,y,\theta) + \frac{\partial}{\partial X}\nabla\phi(x,y,\theta)\cdot (x - x^{(clean)}) \]
\[+ \frac{\partial}{\partial Y}\phi(x,y,\theta)\cdot (y - y^{(clean)})
\]
When we take the expected value:
\[
\mathbb{E}(\nabla\phi(x_{clean},y_{clean},\theta)) \approx \nabla\phi(x,y,\theta) + \frac{\partial}{\partial X}\nabla\phi(x,y,\theta)\cdot \mathbb{E}(\Delta x) \]
\[+ \frac{\partial}{\partial Y}\nabla\phi(x,y,\theta)\cdot \mathbb{E}(\Delta y)
\]
So the resulting estimation formula takes the following form:
\[
\approx \nabla\phi(x,y,\theta) + M_x \cdot \mathbb{E}(\Delta x) + M_y \cdot \mathbb{E}(\Delta y)
\]
Recall that we have a $d$ dimensional feature space and $l$ dimensional label space.
Then, $M_x = \frac{\partial}{\partial X}\nabla\phi$ is an $d \times d$ matrix, and $M_y = \frac{\partial}{\partial Y}\nabla\phi$ is a $d \times l$ matrix.
Both of these matrices are computed with respect to dirty data, and we will present an example.
$\Delta x$ is a $d$ dimensional vector where each component represents a change in that feature and $\Delta y$ is an $l$ dimensional vector that represents the change in each of the labels.

\subsection{Example $M_x$, $M_y$}\label{example-deriv}
\noindent\textbf{Linear Regression: }
\[
\nabla\phi(x,y,\theta) = (\theta^Tx - y)x
\]
For a record, $r$, suppose we have a feature vector $x$.
If we take the partial derivatives with respect to x, $M_x$ is:
\[
M_x[i,i] = 2x[i] + \sum_{i \ne j} \theta[j]x[j] - y 
\]
\[
M_x[i,j] = \theta[j]x[i]
\]
Similarly $M_y$ is:
\[
M_y[i,1] = x[i] 
\]

\vspace{0.5em}

\noindent\textbf{Logistic Regression: } 
\[
\nabla\phi(x,y,\theta) = (h(\theta^Tx) - y)x
\]
where
\[
h(z) = \frac{1}{1+e^{-z}}
\]
we can rewrite this as:
\[
h_{\theta}(x) = \frac{1}{1+e^{\theta^Tx}}
\]
\[
\nabla\phi(x,y,\theta) = (h_{\theta}(x) - y)x
\]
In component form,
\[
g = \nabla\phi(x,y,\theta)
\]
\[
g[i] = h_{\theta}(x)\cdot x[i] - yx[i]
\]
Therefore,
\[
M_x[i,i] = h_{\theta}(x)\cdot(1- h_{\theta}(x))\cdot \theta[i] x[i] + h_{\theta}(x) - y
\]
\[
M_x[i,j] = h_{\theta}(x)\cdot(1- h_{\theta}(x))\cdot \theta[j] x[i] + h_{\theta}(x)
\]
\[
M_y[i,1] = x[i] 
\]

\noindent\textbf{SVM: } 
\[
\nabla\phi(x,y,\theta) =
\begin{cases}      
-y\cdot\boldsymbol{x} \text{ if } y\cdot\boldsymbol{x}\cdot\theta \le 1 \\
0\ \text{ if } y\ \boldsymbol{x}\cdot\theta \geq 1      
\end{cases}
\]
Therefore,
\[
M_x[i,i] = \begin{cases}      
-y[i] \text{ if } y\cdot\boldsymbol{x}\cdot\theta \le 1 \\
0\ \text{ if } y\ \boldsymbol{x}\cdot\theta \geq 1      
\end{cases} 
\]
\[
M_x[i,j] = 0
\]
\[
M_y[i,1] = x[i] 
\]


\subsection{Experimental Comparison}
\subsubsection{Robust Logistic Regression}\label{rlogit}
We use the algorithm from Feng et al. for robust logistic regression.
\begin{enumerate}
\item Input: Contaminated training samples $\{(x_1, y_1), . . . ,(x_{n}
, y_{n})\}$ an upper bound on the number of outliers n, number of inliers n and sample dimension p.
\item Initialization: Set \[T = 4\sqrt{\log p/n + \log n/n}\]
\item Remove samples $(xi
, yi)$ whose magnitude satisfies $\|x_i\| \ge T$.
\item Solve regularized logistic regression problem.
\end{enumerate}

\subsubsection{Active Learning}\label{al}
There is a well established link between Active Learning and Online Gradient-based Algorithms \cite{guillory2009active}. 
To fairly evaluate an Active Learning methodology in these experiments, we run a gradient descent where examples are priortized by expected gradient length (explained in \cite{settles2010active}).
Ignoring the detection step, there are two differences between this algorithm and the one we propose.
First, we calculate the gradient with respect to the an estimate of the clean data, and second we sample rather than using a deterministic ordering.
While such Active Learning algorithms have been studied in the Learning Theory community, they have not been adopted in data cleaning or crowdsourcing research.
Typical algorithms include uncertainty sampling, where a classifier prioritized data closest to the margin.
However, algorithms such as uncertainty sampling focus on the narrow problem of label acquisition in hyperplane classifiers; a problem too narrow for application in this setting.

\subsection{Dollars For Docs Errors and Constraints}\label{dfd-errors}
Example errors include: 

\vspace{0.25em}

\noindent \textbf{Corporations are inconsistently represented: } ``Pfizer", ``Pfizer Inc.", ``Pfizer Incorporated".

\vspace{0.25em}

\noindent \textbf{Drugs are inconsistently represented: } ``TAXOTERE  DOCETAXEL -PROSTATE CANCER" and ``TAXOTERE"

\vspace{0.25em}

\noindent \textbf{Label of covered and not covered are not consistent: } ``No", ``Yes",``N", ``This study is not supported", ``None", ``Combination"

\vspace{0.25em} 

\noindent \textbf{Research subject must be a drug OR a medical device and not both: } ``BIO FLU QPAN H7N9AS03 Vaccine" and ``BIO FLU QPAN H7N9AS03 Device"

\vspace{0.5em} 

We encoded the problems as with the following constraints as data quality rules.

\vspace{0.25em}

\noindent \textbf{Rule 1: } Matching dependency on corporation (Weighted Jaccard Similarity $>$ 0.8).

\vspace{0.25em}

\noindent \textbf{Rule 2: } Matching dependency on drug (Weighted Jaccard Similarity $>$ 0.8).

\vspace{0.25em}

\noindent \textbf{Rule 3: } Label must either be ``covered" or ``not covered".

\vspace{0.25em} 

\noindent \textbf{Rule 4: } Either drug or medical device should be null.

\vspace{0.5em}

\subsection{MNIST Errors}
We include visualization of the errors that we generated for the MNIST experiment.
\begin{figure}[ht]
\centering
\includegraphics[scale=0.20]{exp/original.png}
 \includegraphics[scale=0.20]{exp/5x5removal.png}
 \includegraphics[scale=0.20]{exp/fuzzy.png}
 \caption{We experiment with two forms of corruption in the MNIST image datasets: 5x5 block removal and making the images fuzzy. Image (a) shows an uncorrupted ``9", image (b) shows one corrupted with block removal, and image (c) shows one that is corrupted with fuzziness. \label{mnist-corr}}
\end{figure}

\end{document}